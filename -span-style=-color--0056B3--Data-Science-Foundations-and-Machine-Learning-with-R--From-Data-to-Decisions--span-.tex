% Options for packages loaded elsewhere
\PassOptionsToPackage{unicode}{hyperref}
\PassOptionsToPackage{hyphens}{url}
\PassOptionsToPackage{dvipsnames,svgnames,x11names}{xcolor}
%
\documentclass[
  letterpaper,
]{svmono}

\usepackage{amsmath,amssymb}
\usepackage{iftex}
\ifPDFTeX
  \usepackage[T1]{fontenc}
  \usepackage[utf8]{inputenc}
  \usepackage{textcomp} % provide euro and other symbols
\else % if luatex or xetex
  \usepackage{unicode-math}
  \defaultfontfeatures{Scale=MatchLowercase}
  \defaultfontfeatures[\rmfamily]{Ligatures=TeX,Scale=1}
\fi
\usepackage{lmodern}
\ifPDFTeX\else  
    % xetex/luatex font selection
    \setmainfont[]{Palatino}
    \setmonofont[]{Inconsolata}
\fi
% Use upquote if available, for straight quotes in verbatim environments
\IfFileExists{upquote.sty}{\usepackage{upquote}}{}
\IfFileExists{microtype.sty}{% use microtype if available
  \usepackage[]{microtype}
  \UseMicrotypeSet[protrusion]{basicmath} % disable protrusion for tt fonts
}{}
\makeatletter
\@ifundefined{KOMAClassName}{% if non-KOMA class
  \IfFileExists{parskip.sty}{%
    \usepackage{parskip}
  }{% else
    \setlength{\parindent}{0pt}
    \setlength{\parskip}{6pt plus 2pt minus 1pt}}
}{% if KOMA class
  \KOMAoptions{parskip=half}}
\makeatother
\usepackage{xcolor}
\setlength{\emergencystretch}{3em} % prevent overfull lines
\setcounter{secnumdepth}{5}
% Make \paragraph and \subparagraph free-standing
\makeatletter
\ifx\paragraph\undefined\else
  \let\oldparagraph\paragraph
  \renewcommand{\paragraph}{
    \@ifstar
      \xxxParagraphStar
      \xxxParagraphNoStar
  }
  \newcommand{\xxxParagraphStar}[1]{\oldparagraph*{#1}\mbox{}}
  \newcommand{\xxxParagraphNoStar}[1]{\oldparagraph{#1}\mbox{}}
\fi
\ifx\subparagraph\undefined\else
  \let\oldsubparagraph\subparagraph
  \renewcommand{\subparagraph}{
    \@ifstar
      \xxxSubParagraphStar
      \xxxSubParagraphNoStar
  }
  \newcommand{\xxxSubParagraphStar}[1]{\oldsubparagraph*{#1}\mbox{}}
  \newcommand{\xxxSubParagraphNoStar}[1]{\oldsubparagraph{#1}\mbox{}}
\fi
\makeatother

\usepackage{color}
\usepackage{fancyvrb}
\newcommand{\VerbBar}{|}
\newcommand{\VERB}{\Verb[commandchars=\\\{\}]}
\DefineVerbatimEnvironment{Highlighting}{Verbatim}{commandchars=\\\{\}}
% Add ',fontsize=\small' for more characters per line
\usepackage{framed}
\definecolor{shadecolor}{RGB}{241,243,245}
\newenvironment{Shaded}{\begin{snugshade}}{\end{snugshade}}
\newcommand{\AlertTok}[1]{\textcolor[rgb]{0.68,0.00,0.00}{#1}}
\newcommand{\AnnotationTok}[1]{\textcolor[rgb]{0.37,0.37,0.37}{#1}}
\newcommand{\AttributeTok}[1]{\textcolor[rgb]{0.40,0.45,0.13}{#1}}
\newcommand{\BaseNTok}[1]{\textcolor[rgb]{0.68,0.00,0.00}{#1}}
\newcommand{\BuiltInTok}[1]{\textcolor[rgb]{0.00,0.23,0.31}{#1}}
\newcommand{\CharTok}[1]{\textcolor[rgb]{0.13,0.47,0.30}{#1}}
\newcommand{\CommentTok}[1]{\textcolor[rgb]{0.37,0.37,0.37}{#1}}
\newcommand{\CommentVarTok}[1]{\textcolor[rgb]{0.37,0.37,0.37}{\textit{#1}}}
\newcommand{\ConstantTok}[1]{\textcolor[rgb]{0.56,0.35,0.01}{#1}}
\newcommand{\ControlFlowTok}[1]{\textcolor[rgb]{0.00,0.23,0.31}{\textbf{#1}}}
\newcommand{\DataTypeTok}[1]{\textcolor[rgb]{0.68,0.00,0.00}{#1}}
\newcommand{\DecValTok}[1]{\textcolor[rgb]{0.68,0.00,0.00}{#1}}
\newcommand{\DocumentationTok}[1]{\textcolor[rgb]{0.37,0.37,0.37}{\textit{#1}}}
\newcommand{\ErrorTok}[1]{\textcolor[rgb]{0.68,0.00,0.00}{#1}}
\newcommand{\ExtensionTok}[1]{\textcolor[rgb]{0.00,0.23,0.31}{#1}}
\newcommand{\FloatTok}[1]{\textcolor[rgb]{0.68,0.00,0.00}{#1}}
\newcommand{\FunctionTok}[1]{\textcolor[rgb]{0.28,0.35,0.67}{#1}}
\newcommand{\ImportTok}[1]{\textcolor[rgb]{0.00,0.46,0.62}{#1}}
\newcommand{\InformationTok}[1]{\textcolor[rgb]{0.37,0.37,0.37}{#1}}
\newcommand{\KeywordTok}[1]{\textcolor[rgb]{0.00,0.23,0.31}{\textbf{#1}}}
\newcommand{\NormalTok}[1]{\textcolor[rgb]{0.00,0.23,0.31}{#1}}
\newcommand{\OperatorTok}[1]{\textcolor[rgb]{0.37,0.37,0.37}{#1}}
\newcommand{\OtherTok}[1]{\textcolor[rgb]{0.00,0.23,0.31}{#1}}
\newcommand{\PreprocessorTok}[1]{\textcolor[rgb]{0.68,0.00,0.00}{#1}}
\newcommand{\RegionMarkerTok}[1]{\textcolor[rgb]{0.00,0.23,0.31}{#1}}
\newcommand{\SpecialCharTok}[1]{\textcolor[rgb]{0.37,0.37,0.37}{#1}}
\newcommand{\SpecialStringTok}[1]{\textcolor[rgb]{0.13,0.47,0.30}{#1}}
\newcommand{\StringTok}[1]{\textcolor[rgb]{0.13,0.47,0.30}{#1}}
\newcommand{\VariableTok}[1]{\textcolor[rgb]{0.07,0.07,0.07}{#1}}
\newcommand{\VerbatimStringTok}[1]{\textcolor[rgb]{0.13,0.47,0.30}{#1}}
\newcommand{\WarningTok}[1]{\textcolor[rgb]{0.37,0.37,0.37}{\textit{#1}}}

\providecommand{\tightlist}{%
  \setlength{\itemsep}{0pt}\setlength{\parskip}{0pt}}\usepackage{longtable,booktabs,array}
\usepackage{calc} % for calculating minipage widths
% Correct order of tables after \paragraph or \subparagraph
\usepackage{etoolbox}
\makeatletter
\patchcmd\longtable{\par}{\if@noskipsec\mbox{}\fi\par}{}{}
\makeatother
% Allow footnotes in longtable head/foot
\IfFileExists{footnotehyper.sty}{\usepackage{footnotehyper}}{\usepackage{footnote}}
\makesavenoteenv{longtable}
\usepackage{graphicx}
\makeatletter
\newsavebox\pandoc@box
\newcommand*\pandocbounded[1]{% scales image to fit in text height/width
  \sbox\pandoc@box{#1}%
  \Gscale@div\@tempa{\textheight}{\dimexpr\ht\pandoc@box+\dp\pandoc@box\relax}%
  \Gscale@div\@tempb{\linewidth}{\wd\pandoc@box}%
  \ifdim\@tempb\p@<\@tempa\p@\let\@tempa\@tempb\fi% select the smaller of both
  \ifdim\@tempa\p@<\p@\scalebox{\@tempa}{\usebox\pandoc@box}%
  \else\usebox{\pandoc@box}%
  \fi%
}
% Set default figure placement to htbp
\def\fps@figure{htbp}
\makeatother
% definitions for citeproc citations
\NewDocumentCommand\citeproctext{}{}
\NewDocumentCommand\citeproc{mm}{%
  \begingroup\def\citeproctext{#2}\cite{#1}\endgroup}
\makeatletter
 % allow citations to break across lines
 \let\@cite@ofmt\@firstofone
 % avoid brackets around text for \cite:
 \def\@biblabel#1{}
 \def\@cite#1#2{{#1\if@tempswa , #2\fi}}
\makeatother
\newlength{\cslhangindent}
\setlength{\cslhangindent}{1.5em}
\newlength{\csllabelwidth}
\setlength{\csllabelwidth}{3em}
\newenvironment{CSLReferences}[2] % #1 hanging-indent, #2 entry-spacing
 {\begin{list}{}{%
  \setlength{\itemindent}{0pt}
  \setlength{\leftmargin}{0pt}
  \setlength{\parsep}{0pt}
  % turn on hanging indent if param 1 is 1
  \ifodd #1
   \setlength{\leftmargin}{\cslhangindent}
   \setlength{\itemindent}{-1\cslhangindent}
  \fi
  % set entry spacing
  \setlength{\itemsep}{#2\baselineskip}}}
 {\end{list}}
\usepackage{calc}
\newcommand{\CSLBlock}[1]{\hfill\break\parbox[t]{\linewidth}{\strut\ignorespaces#1\strut}}
\newcommand{\CSLLeftMargin}[1]{\parbox[t]{\csllabelwidth}{\strut#1\strut}}
\newcommand{\CSLRightInline}[1]{\parbox[t]{\linewidth - \csllabelwidth}{\strut#1\strut}}
\newcommand{\CSLIndent}[1]{\hspace{\cslhangindent}#1}

% template.tex: Springer-compatible LaTeX header for Quarto

\usepackage{fontspec}

% Essential Springer-required fonts and packages
% \usepackage{mathptmx} % Times font (Springer default)
% \usepackage{helvet}   % Helvetica font (for sans-serif text)
% \usepackage{courier}  % Courier font (monospaced text)
% \usepackage{type1cm}  % Ensures scalable fonts (Type 1)

% Additional Springer-specific settings:
\usepackage{makeidx}  % Required for indexing
\makeindex            % Enable index generation

% For better tables
\usepackage{booktabs}

% Figures and graphics settings
\usepackage{graphicx}

% Math packages (often used in Springer books)
\usepackage{amsmath}
\usepackage{amssymb}

% Springer-specific gray boxes (optional):
\usepackage{tcolorbox}
\tcbuselibrary{listingsutf8}
\tcbset{
  boxrule=0pt,
  colback=gray!10,
  colframe=gray!40,
  sharp corners
}

\usepackage{listings}
\lstset{
  breaklines=true,
  breakatwhitespace=false,
  postbreak=\mbox{\textcolor{gray}{$\hookrightarrow$}\space},
  basicstyle=\monofont,
  columns=fullflexible,
  keepspaces=true,
  frame=single,
  xleftmargin=1em,
  tabsize=2
}


\usepackage{fvextra}
\DefineVerbatimEnvironment{Highlighting}{Verbatim}{
  breaklines=true,
  breakanywhere=true,
  fontsize=\small,
  commandchars=\\\{\}
}

\numberwithin{example}{subsection}


% Custom commands (optional):
\newcommand{\R}{\textsf{R}}

% Epigraph environment
\newenvironment{chapterquote}
  {\begin{quote}\itshape}
  {\end{quote}\vspace{2em}}

\usepackage{booktabs}
\usepackage{longtable}
\usepackage{array}
\usepackage{multirow}
\usepackage{wrapfig}
\usepackage{float}
\usepackage{colortbl}
\usepackage{pdflscape}
\usepackage{tabu}
\usepackage{threeparttable}
\usepackage{threeparttablex}
\usepackage[normalem]{ulem}
\usepackage{makecell}
\usepackage{xcolor}
\makeatletter
\@ifpackageloaded{bookmark}{}{\usepackage{bookmark}}
\makeatother
\makeatletter
\@ifpackageloaded{caption}{}{\usepackage{caption}}
\AtBeginDocument{%
\ifdefined\contentsname
  \renewcommand*\contentsname{Table of contents}
\else
  \newcommand\contentsname{Table of contents}
\fi
\ifdefined\listfigurename
  \renewcommand*\listfigurename{List of Figures}
\else
  \newcommand\listfigurename{List of Figures}
\fi
\ifdefined\listtablename
  \renewcommand*\listtablename{List of Tables}
\else
  \newcommand\listtablename{List of Tables}
\fi
\ifdefined\figurename
  \renewcommand*\figurename{Figure}
\else
  \newcommand\figurename{Figure}
\fi
\ifdefined\tablename
  \renewcommand*\tablename{Table}
\else
  \newcommand\tablename{Table}
\fi
}
\@ifpackageloaded{float}{}{\usepackage{float}}
\floatstyle{ruled}
\@ifundefined{c@chapter}{\newfloat{codelisting}{h}{lop}}{\newfloat{codelisting}{h}{lop}[chapter]}
\floatname{codelisting}{Listing}
\newcommand*\listoflistings{\listof{codelisting}{List of Listings}}
\makeatother
\makeatletter
\makeatother
\makeatletter
\@ifpackageloaded{caption}{}{\usepackage{caption}}
\@ifpackageloaded{subcaption}{}{\usepackage{subcaption}}
\makeatother

\usepackage{bookmark}

\IfFileExists{xurl.sty}{\usepackage{xurl}}{} % add URL line breaks if available
\urlstyle{same} % disable monospaced font for URLs
\hypersetup{
  pdftitle={Data Science Foundations and Machine Learning with R: From Data to Decisions},
  pdfauthor={Reza Mohammadi},
  colorlinks=true,
  linkcolor={blue},
  filecolor={Maroon},
  citecolor={Blue},
  urlcolor={Blue},
  pdfcreator={LaTeX via pandoc}}


\title{{Data Science Foundations and Machine Learning with R: From Data
to Decisions}}
\author{\href{https://www.uva.nl/profile/a.mohammadi}{{Reza Mohammadi}}}
\date{27 November 2025}

\begin{document}
\maketitle

\renewcommand*\contentsname{Table of contents}
{
\hypersetup{linkcolor=}
\setcounter{tocdepth}{2}
\tableofcontents
}

\bookmarksetup{startatroot}

\chapter*{}\label{section}
\addcontentsline{toc}{chapter}{}

\markboth{}{}

\bookmarksetup{startatroot}

\chapter{Exploratory Data Analysis}\label{sec-ch4-EDA}

\begin{chapterquote}
The greatest value of a picture is when it forces us to notice what we never expected.

\hfill — John Tukey
\end{chapterquote}

Exploratory Data Analysis (EDA) is the essential first step before
building models or conducting statistical inference. It involves
examining data carefully, thoroughly, and creatively to uncover
insights. By revealing unexpected patterns, identifying anomalies, and
highlighting potential relationships, EDA shapes the direction of all
subsequent analysis.

EDA plays a pivotal role in the Data Science Workflow (see
\textbf{?@fig-ch2\_DSW}), serving as the bridge between Data Preparation
(Chapter \textbf{?@sec-ch3-data-preparation}) and Data Setup to Model
(Chapter \textbf{?@sec-ch6-setup-data}). This stage deepens our
understanding of the data's structure, quality, and potential, ensuring
that downstream decisions rest on a solid empirical foundation.

Unlike formal hypothesis testing, EDA is not rigid or rule-driven. It is
an iterative, open-ended process that encourages curiosity and
experimentation. Different datasets raise different questions, and some
exploratory paths will reveal meaningful trends while others uncover
data issues or lead to dead ends. Through this process, analysts develop
intuition, refine their focus, and identify the most informative
features for modelling.

The purpose of EDA is not to confirm theories but to generate insight.
Summary statistics, exploratory visualisations, and correlation measures
provide an initial map of the data landscape. These findings should be
interpreted cautiously, as early patterns may not represent causal
relationships. In Chapter \textbf{?@sec-ch5-statistics}, we introduce
formal tools for statistical inference that build on this exploratory
foundation.

EDA also highlights the importance of practical relevance. In large
datasets, weak patterns can easily reach statistical significance yet
offer little real-world value. For example, a slight difference in
customer engagement may be statistically detectable but too small to
influence business decisions. Integrating domain expertise is therefore
essential when interpreting exploratory findings.

Finally, EDA is central to assessing and improving data quality.
Outliers, missing values, inconsistent formats, and redundant variables
often emerge during exploration. Addressing these issues early ensures
that later models are both reliable and interpretable. The choice of EDA
techniques depends on the nature of the data and the analytical
questions at hand. Histograms and box plots reveal distributions, while
scatter plots and correlation matrices expose relationships. The next
sections introduce these tools in context and explain how to apply them
effectively.

\subsection*{What This Chapter Covers}\label{what-this-chapter-covers}
\addcontentsline{toc}{subsection}{What This Chapter Covers}

This chapter introduces exploratory data analysis as a critical stage in
the data science workflow. You will learn how to use summary statistics
and visual techniques to examine variable distributions, detect
anomalies, and uncover relationships that inform downstream modelling.
The chapter also shows how correlation analysis helps identify
redundancy and how multivariate exploration can reveal patterns that
enhance predictive insight.

The chapter begins with \emph{EDA as Data Storytelling}, which
emphasises the importance of communicating exploratory findings with
clarity and context. This is followed by \emph{Key Objectives and
Guiding Questions for EDA}, which outline the main aims of exploration
and the questions that support a structured analytical process.

Building on these ideas, the chapter presents a detailed exploration of
the \emph{churnCredit} dataset from the \textbf{liver} package. This
example illustrates how real-world patterns emerge from data, how
visualisations illuminate customer behaviour, and how exploratory
insights prepare the ground for classification modelling using k-nearest
neighbours in Chapter \textbf{?@sec-ch7-classification-knn}.

The chapter concludes with a comprehensive set of exercises and hands-on
projects using two additional real-world datasets (\emph{bank} and
\emph{churn}, also from the \textbf{liver} package). These activities
provide further practice with EDA techniques and lay the foundation for
the neural network case study in Chapter
\textbf{?@sec-ch12-neural-networks}.

\section{EDA as Data Storytelling}\label{eda-as-data-storytelling}

Exploratory data analysis is not only a technical process for uncovering
patterns; it is also a way of communicating insights clearly and
persuasively. While EDA reveals structure, anomalies, and relationships,
these findings gain value only when they are presented with context and
purpose. Data storytelling plays a central role in this process by
transforming raw exploration into insight.

Effective storytelling in data science weaves together analytical
evidence, contextual knowledge, and visual clarity. Rather than
presenting statistics or plots in isolation, strong analysis connects
each observation to a broader narrative. Whether the audience includes
analysts, business stakeholders, or policymakers, the goal is to convey
findings in a way that is meaningful and relevant.

Consider a typical observation: customers with high daytime usage appear
more likely to churn. Stating this pattern is informative, but it does
not yet offer understanding. A narrative that links the pattern to its
implications brings the analysis to life:

\emph{``Customers with extensive daytime usage show a higher tendency to
churn, possibly due to pricing concerns or dissatisfaction with service
quality. Targeted retention strategies, such as customised discounts or
more flexible pricing plans, may help address this risk.''}

This shift from description to interpretation is at the heart of data
storytelling. It invites reflection and supports informed
decision-making.

Visualisation is central to this process. While summary statistics offer
a structural overview, visual displays make patterns tangible. Scatter
plots and correlation matrices highlight relationships among numerical
features; histograms and box plots clarify distributions and skewness;
bar charts and mosaic visualisations reveal differences across
categories. Choosing appropriate visual tools not only strengthens
analysis but also improves communication.

Storytelling through data is widely used across domains, from business
and journalism to public policy and scientific research. A well-known
example is Hans Rosling's TED Talk
\href{https://www.ted.com/talks/hans_rosling_new_insights_on_poverty}{\emph{New
insights on poverty}}, where decades of demographic and economic data
are presented in an engaging, intuitive format. Figure
Figure~\ref{fig-EDA-fig-1}, adapted from his presentation, illustrates
how GDP per capita and life expectancy have changed across world regions
from 1950 to 2019. The figure is generated from the \emph{gapminder}
dataset available in the \textbf{liver} package and visualised using
\textbf{ggplot2}. Although drawn from global development, the same
principles apply when exploring customer behaviour, financial trends, or
service outcomes.

\begin{figure}[H]

\centering{

\includegraphics[width=1\linewidth,height=\textheight,keepaspectratio]{4-Exploratory-data-analysis_files/figure-pdf/fig-EDA-fig-1-1.pdf}

}

\caption{\label{fig-EDA-fig-1}Changes in GDP per capita and life
expectancy by region from 1950 to 2019. Dot size is proportional to
population.}

\end{figure}%

As you conduct EDA, it is useful to ask not only \emph{what} the data
shows, but also \emph{why} those patterns matter. What story is
emerging? How might that story inform a decision, challenge an
assumption, or motivate further analysis? Thinking in narrative terms
ensures that exploratory work is not merely descriptive but purposeful,
rooted in the real-world questions that prompted the analysis.

The next section builds on these ideas by outlining the key objectives
and guiding questions that shape effective exploratory analysis.
Together, they provide a structured yet flexible foundation for the
detailed EDA of customer churn that follows.

\section{Objectives and Guiding Questions for
EDA}\label{sec-EDA-objectives-questions}

EDA marks the first substantive interaction between analyst and dataset,
the moment when raw information begins to reveal its structure,
surprises, and potential narratives. Rather than moving directly into
modelling, experienced analysts pause to ask what the data contains,
which patterns stand out, and which issues require attention.

A useful starting point is to clarify what exploratory analysis is
designed to accomplish. At its core, EDA seeks to understand the
structure of the data, including variable types, value ranges, missing
entries, and possible anomalies. It examines how individual variables
are distributed, identifying central tendencies, variation, and
skewness. It investigates how variables relate to one another, revealing
associations, dependencies, or interactions that may later contribute to
predictive models. It also detects patterns and outliers that might
indicate errors, unusual subgroups, or emerging signals worth
investigating further.

These objectives form the foundation for effective modelling. They help
analysts refine which features deserve emphasis, anticipate potential
challenges, and identify early insights that can guide the direction of
later stages in the workflow.

Exploration becomes more productive when guided by focused questions.
These questions can be grouped broadly into those concerning individual
variables and those concerning relationships among variables. When
examining variables one at a time, the guiding questions ask what each
variable reveals on its own, how it is distributed, whether missing
values follow a particular pattern, and whether any irregularities stand
out. Histograms, box plots, and summary statistics are familiar tools
for answering such questions.

When shifting to relationships among variables, the focus moves to how
predictors relate to the target, whether any variables are strongly
correlated, whether redundancies or interactions might influence
modelling, and how categorical and numerical variables combine to reveal
structure. Scatter plots, grouped visualisations, and correlation
matrices help reveal these patterns and support thoughtful feature
selection.

A recurring challenge, especially for students, is choosing which plots
or techniques best suit different types of data. Table
\ref{tbl-EDA-table-tools} summarises commonly used exploratory
objectives alongside appropriate analytical tools. It serves as a
practical reference when deciding how to approach unfamiliar datasets or
new analytical questions.

\begin{table}

\caption{\label{tbl-EDA-table-tools}Overview of Recommended Tools for
Common EDA Objectives.}

\centering{

\centering
\begin{tabular}[t]{>{\raggedright\arraybackslash}p{12em}>{\raggedright\arraybackslash}p{11em}>{\raggedright\arraybackslash}p{13em}}
\toprule
Exploratory.Objective & Applicable.Data.Type & Recommended.Techniques\\
\midrule
Examine a variable’s distribution & Numerical & Histogram, box plot, density plot, summary statistics\\
Summarize a categorical variable & Categorical & Bar chart, frequency table\\
Identify outliers & Numerical & Box plot, histogram\\
Detect missing data patterns & Any & Summary statistics, missingness maps\\
Explore the relationship between two numerical variables & Numerical \& Numerical & Scatter plot, correlation coefficient\\
\addlinespace
Compare a numerical variable across groups & Numerical \& Categorical & Box plot, grouped bar chart, violin plot\\
Analyze interactions between two categorical variables & Categorical \& Categorical & Stacked bar chart, mosaic plot, contingency table\\
Assess correlation among multiple numerical variables & Multiple Numerical & Correlation matrix, scatterplot matrix\\
\bottomrule
\end{tabular}

}

\end{table}%

By aligning objectives with guiding questions and appropriate methods,
EDA becomes more than a routine diagnostic stage. It becomes a strategic
component of the workflow that enhances data quality, informs feature
construction, and lays the groundwork for effective modelling.

The next section applies these principles through a detailed EDA of
customer churn, showing how statistical summaries, visual tools, and
domain understanding can uncover patterns that support predictive
analysis.

\section{\texorpdfstring{EDA in Practice: The \emph{churnCredit}
Dataset}{EDA in Practice: The churnCredit Dataset}}\label{sec-ch4-EDA-churn}

Exploratory data analysis (EDA) is most meaningful when applied to real
data and practical questions. In this section, we illustrate the process
using the \emph{churnCredit} dataset, which contains demographic,
behavioral, and financial information about customers, along with a
binary variable indicating whether each customer has churned---that is,
discontinued the service.

This walkthrough follows the structure of the Data Science Workflow
introduced in Chapter \textbf{?@sec-ch2-intro-data-science}. We begin by
revisiting the first two steps, \emph{Problem Understanding} and
\emph{Data Preparation}, to establish the business context and examine
the dataset's structure. The main emphasis is on \emph{Step 3:
Exploratory Data Analysis}, where visualizations, summary statistics,
and guiding questions are used to uncover meaningful patterns related to
customer churn.

The insights developed in this section provide a foundation for the
subsequent stages of analysis: preparing the data for modeling in
Chapter \textbf{?@sec-ch6-setup-data}, constructing predictive models
using k-nearest neighbors in Chapter
\textbf{?@sec-ch7-classification-knn}, and assessing model performance
in Chapter \textbf{?@sec-ch8-evaluation}. Working through these stages
in sequence demonstrates how a thorough exploratory analysis enhances
understanding and strengthens the decisions made through modeling.

\subsection*{\texorpdfstring{Problem Understanding for the
\emph{churnCredit}
Dataset}{Problem Understanding for the churnCredit Dataset}}\label{problem-understanding-for-the-churncredit-dataset}
\addcontentsline{toc}{subsection}{Problem Understanding for the
\emph{churnCredit} Dataset}

A manager at a bank has become increasingly concerned about the growing
number of customers closing their credit card accounts. Understanding
why customers leave---and being able to anticipate who is at risk of
doing so---has become a strategic priority. Predicting churn would allow
the bank to intervene proactively, offering improved services or
incentives to retain valuable customers.

Customer churn, the loss of existing clients, is a persistent challenge
in subscription-based industries such as banking, telecommunications,
and streaming services. Because retaining an existing customer is
typically more cost-effective than acquiring a new one, identifying the
factors that drive churn is a central task for both analysts and
decision-makers.

From a business perspective, this problem gives rise to three key
questions:

\begin{itemize}
\item
  \emph{Why} are customers choosing to leave?
\item
  \emph{What} behavioral or demographic characteristics are associated
  with higher churn risk?
\item
  \emph{How} can these insights guide strategies for improving customer
  retention?
\end{itemize}

Exploratory data analysis provides a foundation for addressing these
questions. By identifying patterns, anomalies, and relationships in the
data, EDA uncovers potential signals that can inform targeted retention
initiatives. It also clarifies how customer attributes and behaviors
interact---insights that later support predictive modeling.

In Chapter \textbf{?@sec-ch7-classification-knn}, we will develop a
k-nearest neighbors (kNN) model to predict customer churn. Before
building that model, however, it is crucial to understand the structure
of the dataset, the nature of its variables, and the relationships they
reveal. The next step is to examine the \emph{churnCredit} dataset in
detail, gaining an understanding of its structure, variables, and the
types of information it provides about customer behavior and churn.

\subsection*{\texorpdfstring{Overview of the \emph{churnCredit}
Dataset}{Overview of the churnCredit Dataset}}\label{overview-of-the-churncredit-dataset}
\addcontentsline{toc}{subsection}{Overview of the \emph{churnCredit}
Dataset}

Before conducting visual or statistical exploration, it is essential to
understand the dataset used throughout this chapter. The
\emph{churnCredit} dataset, available in the \textbf{liver} package,
serves as a realistic case study for applying exploratory data analysis.
It contains over 10,000 customer records and 21 variables combining
demographic information, account characteristics, credit usage, and
customer interaction metrics.

The key variable of interest is \texttt{churn}, which indicates whether
a customer has closed their credit card account (\texttt{yes}) or
remained active (\texttt{no}). This binary outcome will later serve as
the target variable for classification modeling in Chapter
\textbf{?@sec-ch7-classification-knn}. At this stage, our objective is
to understand the structure, content, and quality of the data that
surround this outcome.

To load and inspect the dataset, run the following commands in R:

\begin{Shaded}
\begin{Highlighting}[]
\FunctionTok{library}\NormalTok{(liver)}

\FunctionTok{data}\NormalTok{(churnCredit)}
\FunctionTok{str}\NormalTok{(churnCredit)}
   \StringTok{\textquotesingle{}data.frame\textquotesingle{}}\SpecialCharTok{:}    \DecValTok{10127}\NormalTok{ obs. of  }\DecValTok{21}\NormalTok{ variables}\SpecialCharTok{:}
    \ErrorTok{$}\NormalTok{ customer.ID          }\SpecialCharTok{:}\NormalTok{ int  }\DecValTok{768805383} \DecValTok{818770008} \DecValTok{713982108} \DecValTok{769911858} \DecValTok{709106358} \DecValTok{713061558} \DecValTok{810347208} \DecValTok{818906208} \DecValTok{710930508} \DecValTok{719661558}\NormalTok{ ...}
    \SpecialCharTok{$}\NormalTok{ age                  }\SpecialCharTok{:}\NormalTok{ int  }\DecValTok{45} \DecValTok{49} \DecValTok{51} \DecValTok{40} \DecValTok{40} \DecValTok{44} \DecValTok{51} \DecValTok{32} \DecValTok{37} \DecValTok{48}\NormalTok{ ...}
    \SpecialCharTok{$}\NormalTok{ gender               }\SpecialCharTok{:}\NormalTok{ Factor w}\SpecialCharTok{/} \DecValTok{2}\NormalTok{ levels }\StringTok{"female"}\NormalTok{,}\StringTok{"male"}\SpecialCharTok{:} \DecValTok{2} \DecValTok{1} \DecValTok{2} \DecValTok{1} \DecValTok{2} \DecValTok{2} \DecValTok{2} \DecValTok{2} \DecValTok{2} \DecValTok{2}\NormalTok{ ...}
    \SpecialCharTok{$}\NormalTok{ education            }\SpecialCharTok{:}\NormalTok{ Factor w}\SpecialCharTok{/} \DecValTok{7}\NormalTok{ levels }\StringTok{"uneducated"}\NormalTok{,}\StringTok{"highschool"}\NormalTok{,..}\SpecialCharTok{:} \DecValTok{2} \DecValTok{4} \DecValTok{4} \DecValTok{2} \DecValTok{1} \DecValTok{4} \DecValTok{7} \DecValTok{2} \DecValTok{1} \DecValTok{4}\NormalTok{ ...}
    \SpecialCharTok{$}\NormalTok{ marital              }\SpecialCharTok{:}\NormalTok{ Factor w}\SpecialCharTok{/} \DecValTok{4}\NormalTok{ levels }\StringTok{"married"}\NormalTok{,}\StringTok{"single"}\NormalTok{,..}\SpecialCharTok{:} \DecValTok{1} \DecValTok{2} \DecValTok{1} \DecValTok{4} \DecValTok{1} \DecValTok{1} \DecValTok{1} \DecValTok{4} \DecValTok{2} \DecValTok{2}\NormalTok{ ...}
    \SpecialCharTok{$}\NormalTok{ income               }\SpecialCharTok{:}\NormalTok{ Factor w}\SpecialCharTok{/} \DecValTok{6}\NormalTok{ levels }\StringTok{"\textless{}40K"}\NormalTok{,}\StringTok{"40K{-}60K"}\NormalTok{,..}\SpecialCharTok{:} \DecValTok{3} \DecValTok{1} \DecValTok{4} \DecValTok{1} \DecValTok{3} \DecValTok{2} \DecValTok{5} \DecValTok{3} \DecValTok{3} \DecValTok{4}\NormalTok{ ...}
    \SpecialCharTok{$}\NormalTok{ card.category        }\SpecialCharTok{:}\NormalTok{ Factor w}\SpecialCharTok{/} \DecValTok{4}\NormalTok{ levels }\StringTok{"blue"}\NormalTok{,}\StringTok{"silver"}\NormalTok{,..}\SpecialCharTok{:} \DecValTok{1} \DecValTok{1} \DecValTok{1} \DecValTok{1} \DecValTok{1} \DecValTok{1} \DecValTok{3} \DecValTok{2} \DecValTok{1} \DecValTok{1}\NormalTok{ ...}
    \SpecialCharTok{$}\NormalTok{ dependent.count      }\SpecialCharTok{:}\NormalTok{ int  }\DecValTok{3} \DecValTok{5} \DecValTok{3} \DecValTok{4} \DecValTok{3} \DecValTok{2} \DecValTok{4} \DecValTok{0} \DecValTok{3} \DecValTok{2}\NormalTok{ ...}
    \SpecialCharTok{$}\NormalTok{ months.on.book       }\SpecialCharTok{:}\NormalTok{ int  }\DecValTok{39} \DecValTok{44} \DecValTok{36} \DecValTok{34} \DecValTok{21} \DecValTok{36} \DecValTok{46} \DecValTok{27} \DecValTok{36} \DecValTok{36}\NormalTok{ ...}
    \SpecialCharTok{$}\NormalTok{ relationship.count   }\SpecialCharTok{:}\NormalTok{ int  }\DecValTok{5} \DecValTok{6} \DecValTok{4} \DecValTok{3} \DecValTok{5} \DecValTok{3} \DecValTok{6} \DecValTok{2} \DecValTok{5} \DecValTok{6}\NormalTok{ ...}
    \SpecialCharTok{$}\NormalTok{ months.inactive      }\SpecialCharTok{:}\NormalTok{ int  }\DecValTok{1} \DecValTok{1} \DecValTok{1} \DecValTok{4} \DecValTok{1} \DecValTok{1} \DecValTok{1} \DecValTok{2} \DecValTok{2} \DecValTok{3}\NormalTok{ ...}
    \SpecialCharTok{$}\NormalTok{ contacts.count}\FloatTok{.12}    \SpecialCharTok{:}\NormalTok{ int  }\DecValTok{3} \DecValTok{2} \DecValTok{0} \DecValTok{1} \DecValTok{0} \DecValTok{2} \DecValTok{3} \DecValTok{2} \DecValTok{0} \DecValTok{3}\NormalTok{ ...}
    \SpecialCharTok{$}\NormalTok{ credit.limit         }\SpecialCharTok{:}\NormalTok{ num  }\DecValTok{12691} \DecValTok{8256} \DecValTok{3418} \DecValTok{3313} \DecValTok{4716}\NormalTok{ ...}
    \SpecialCharTok{$}\NormalTok{ revolving.balance    }\SpecialCharTok{:}\NormalTok{ int  }\DecValTok{777} \DecValTok{864} \DecValTok{0} \DecValTok{2517} \DecValTok{0} \DecValTok{1247} \DecValTok{2264} \DecValTok{1396} \DecValTok{2517} \DecValTok{1677}\NormalTok{ ...}
    \SpecialCharTok{$}\NormalTok{ available.credit     }\SpecialCharTok{:}\NormalTok{ num  }\DecValTok{11914} \DecValTok{7392} \DecValTok{3418} \DecValTok{796} \DecValTok{4716}\NormalTok{ ...}
    \SpecialCharTok{$}\NormalTok{ transaction.amount}\FloatTok{.12}\SpecialCharTok{:}\NormalTok{ int  }\DecValTok{1144} \DecValTok{1291} \DecValTok{1887} \DecValTok{1171} \DecValTok{816} \DecValTok{1088} \DecValTok{1330} \DecValTok{1538} \DecValTok{1350} \DecValTok{1441}\NormalTok{ ...}
    \SpecialCharTok{$}\NormalTok{ transaction.count}\FloatTok{.12} \SpecialCharTok{:}\NormalTok{ int  }\DecValTok{42} \DecValTok{33} \DecValTok{20} \DecValTok{20} \DecValTok{28} \DecValTok{24} \DecValTok{31} \DecValTok{36} \DecValTok{24} \DecValTok{32}\NormalTok{ ...}
    \SpecialCharTok{$}\NormalTok{ ratio.amount.Q4.Q1   }\SpecialCharTok{:}\NormalTok{ num  }\FloatTok{1.33} \FloatTok{1.54} \FloatTok{2.59} \FloatTok{1.41} \FloatTok{2.17}\NormalTok{ ...}
    \SpecialCharTok{$}\NormalTok{ ratio.count.Q4.Q1    }\SpecialCharTok{:}\NormalTok{ num  }\FloatTok{1.62} \FloatTok{3.71} \FloatTok{2.33} \FloatTok{2.33} \FloatTok{2.5}\NormalTok{ ...}
    \SpecialCharTok{$}\NormalTok{ utilization.ratio    }\SpecialCharTok{:}\NormalTok{ num  }\FloatTok{0.061} \FloatTok{0.105} \DecValTok{0} \FloatTok{0.76} \DecValTok{0} \FloatTok{0.311} \FloatTok{0.066} \FloatTok{0.048} \FloatTok{0.113} \FloatTok{0.144}\NormalTok{ ...}
    \SpecialCharTok{$}\NormalTok{ churn                }\SpecialCharTok{:}\NormalTok{ Factor w}\SpecialCharTok{/} \DecValTok{2}\NormalTok{ levels }\StringTok{"yes"}\NormalTok{,}\StringTok{"no"}\SpecialCharTok{:} \DecValTok{2} \DecValTok{2} \DecValTok{2} \DecValTok{2} \DecValTok{2} \DecValTok{2} \DecValTok{2} \DecValTok{2} \DecValTok{2} \DecValTok{2}\NormalTok{ ...}
\end{Highlighting}
\end{Shaded}

The dataset is stored as a \texttt{data.frame} with 10127 observations
and 21 variables. The predictors include both numerical and categorical
features describing customer demographics, spending behavior, credit
management, and engagement with the bank. Among the 21 variables, eight
are categorical (\texttt{gender}, \texttt{education}, \texttt{marital},
\texttt{income}, \texttt{card.category}, \texttt{churn}, and two
identifiers related to grouping), while the remaining features are
numerical. The categorical variables capture demographic or qualitative
groupings, whereas the numerical features represent continuous measures
such as credit limits, transaction amounts, and utilization ratios. This
distinction will guide the choice of visualization and summary
techniques in the following sections.

A structured overview of the variables is provided below:

\begin{itemize}
\tightlist
\item
  \texttt{customer.ID}: Unique identifier for each account holder.
\item
  \texttt{age}: Age of the customer, in years.
\item
  \texttt{gender}: Gender of the account holder.
\item
  \texttt{education}: Educational qualification (high-school, college,
  graduate, uneducated, post-graduate, doctorate, unknown).
\item
  \texttt{marital}: Marital status (married, single, divorced, unknown).
\item
  \texttt{income}: Annual income bracket (less than \texttt{\$40K},
  \texttt{\$40K–\$60K}, \texttt{\$60K–\$80K}, \texttt{\$80K–\$120K},
  over \texttt{\$120K}, unknown).
\item
  \texttt{card.category}: Credit card type (blue, silver, gold,
  platinum).
\item
  \texttt{dependent.count}: Number of dependents.
\item
  \texttt{months.on.book}: Tenure with the bank, in months.
\item
  \texttt{relationship.count}: Total number of products held by the
  customer (1--6).
\item
  \texttt{months.inactive}: Number of inactive months in the past 12
  months.
\item
  \texttt{contacts.count.12}: Number of customer service contacts in the
  past 12 months.
\item
  \texttt{credit.limit}: Total credit card limit.
\item
  \texttt{revolving.balance}: Current revolving balance on the credit
  card.
\item
  \texttt{available.credit}: Available credit line, representing the
  unused portion of the credit limit. Calculated as
  \texttt{credit.limit\ -\ revolving.balance}.
\item
  \texttt{transaction.amount.12}: Total transaction amount in the past
  12 months.
\item
  \texttt{transaction.count.12}: Total number of transactions in the
  past 12 months.
\item
  \texttt{ratio.amount.Q4.Q1}: Ratio of total transaction amount in the
  fourth quarter to that in the first quarter.
\item
  \texttt{ratio.count.Q4.Q1}: Ratio of total transaction count in the
  fourth quarter to that in the first quarter.
\item
  \texttt{utilization.ratio}: Average credit utilization ratio, defined
  as \texttt{revolving.balance\ /\ credit.limit}.
\item
  \texttt{churn}: Indicator of whether the account was closed
  (\texttt{yes}) or remained active (\texttt{no}).
\end{itemize}

To obtain an initial overview of variable distributions and potential
irregularities, we can use the \texttt{summary()} function:

\begin{Shaded}
\begin{Highlighting}[]
\FunctionTok{summary}\NormalTok{(churnCredit)}
\NormalTok{     customer.ID             age           gender             education   }
\NormalTok{    Min.   }\SpecialCharTok{:}\DecValTok{708082083}\NormalTok{   Min.   }\SpecialCharTok{:}\FloatTok{26.00}\NormalTok{   female}\SpecialCharTok{:}\DecValTok{5358}\NormalTok{   uneducated   }\SpecialCharTok{:}\DecValTok{1487}  
    \DecValTok{1}\NormalTok{st Qu.}\SpecialCharTok{:}\DecValTok{713036770}   \DecValTok{1}\NormalTok{st Qu.}\SpecialCharTok{:}\FloatTok{41.00}\NormalTok{   male  }\SpecialCharTok{:}\DecValTok{4769}\NormalTok{   highschool   }\SpecialCharTok{:}\DecValTok{2013}  
\NormalTok{    Median }\SpecialCharTok{:}\DecValTok{717926358}\NormalTok{   Median }\SpecialCharTok{:}\FloatTok{46.00}\NormalTok{                 college      }\SpecialCharTok{:}\DecValTok{1013}  
\NormalTok{    Mean   }\SpecialCharTok{:}\DecValTok{739177606}\NormalTok{   Mean   }\SpecialCharTok{:}\FloatTok{46.33}\NormalTok{                 graduate     }\SpecialCharTok{:}\DecValTok{3128}  
    \DecValTok{3}\NormalTok{rd Qu.}\SpecialCharTok{:}\DecValTok{773143533}   \DecValTok{3}\NormalTok{rd Qu.}\SpecialCharTok{:}\FloatTok{52.00}\NormalTok{                 post}\SpecialCharTok{{-}}\NormalTok{graduate}\SpecialCharTok{:} \DecValTok{516}  
\NormalTok{    Max.   }\SpecialCharTok{:}\DecValTok{828343083}\NormalTok{   Max.   }\SpecialCharTok{:}\FloatTok{73.00}\NormalTok{                 doctorate    }\SpecialCharTok{:} \DecValTok{451}  
\NormalTok{                                                      unknown      }\SpecialCharTok{:}\DecValTok{1519}  
\NormalTok{        marital          income      card.category  dependent.count}
\NormalTok{    married }\SpecialCharTok{:}\DecValTok{4687}   \SpecialCharTok{\textless{}}\DecValTok{40}\NormalTok{K    }\SpecialCharTok{:}\DecValTok{3561}\NormalTok{   blue    }\SpecialCharTok{:}\DecValTok{9436}\NormalTok{   Min.   }\SpecialCharTok{:}\FloatTok{0.000}  
\NormalTok{    single  }\SpecialCharTok{:}\DecValTok{3943}   \DecValTok{40}\NormalTok{K}\DecValTok{{-}60}\NormalTok{K }\SpecialCharTok{:}\DecValTok{1790}\NormalTok{   silver  }\SpecialCharTok{:} \DecValTok{555}   \DecValTok{1}\NormalTok{st Qu.}\SpecialCharTok{:}\FloatTok{1.000}  
\NormalTok{    divorced}\SpecialCharTok{:} \DecValTok{748}   \DecValTok{60}\NormalTok{K}\DecValTok{{-}80}\NormalTok{K }\SpecialCharTok{:}\DecValTok{1402}\NormalTok{   gold    }\SpecialCharTok{:} \DecValTok{116}\NormalTok{   Median }\SpecialCharTok{:}\FloatTok{2.000}  
\NormalTok{    unknown }\SpecialCharTok{:} \DecValTok{749}   \DecValTok{80}\NormalTok{K}\DecValTok{{-}120}\NormalTok{K}\SpecialCharTok{:}\DecValTok{1535}\NormalTok{   platinum}\SpecialCharTok{:}  \DecValTok{20}\NormalTok{   Mean   }\SpecialCharTok{:}\FloatTok{2.346}  
                    \SpecialCharTok{\textgreater{}}\DecValTok{120}\NormalTok{K   }\SpecialCharTok{:} \DecValTok{727}                   \DecValTok{3}\NormalTok{rd Qu.}\SpecialCharTok{:}\FloatTok{3.000}  
\NormalTok{                    unknown }\SpecialCharTok{:}\DecValTok{1112}\NormalTok{                   Max.   }\SpecialCharTok{:}\FloatTok{5.000}  
                                                                   
\NormalTok{    months.on.book  relationship.count months.inactive contacts.count}\FloatTok{.12}
\NormalTok{    Min.   }\SpecialCharTok{:}\FloatTok{13.00}\NormalTok{   Min.   }\SpecialCharTok{:}\FloatTok{1.000}\NormalTok{      Min.   }\SpecialCharTok{:}\FloatTok{0.000}\NormalTok{   Min.   }\SpecialCharTok{:}\FloatTok{0.000}    
    \DecValTok{1}\NormalTok{st Qu.}\SpecialCharTok{:}\FloatTok{31.00}   \DecValTok{1}\NormalTok{st Qu.}\SpecialCharTok{:}\FloatTok{3.000}      \DecValTok{1}\NormalTok{st Qu.}\SpecialCharTok{:}\FloatTok{2.000}   \DecValTok{1}\NormalTok{st Qu.}\SpecialCharTok{:}\FloatTok{2.000}    
\NormalTok{    Median }\SpecialCharTok{:}\FloatTok{36.00}\NormalTok{   Median }\SpecialCharTok{:}\FloatTok{4.000}\NormalTok{      Median }\SpecialCharTok{:}\FloatTok{2.000}\NormalTok{   Median }\SpecialCharTok{:}\FloatTok{2.000}    
\NormalTok{    Mean   }\SpecialCharTok{:}\FloatTok{35.93}\NormalTok{   Mean   }\SpecialCharTok{:}\FloatTok{3.813}\NormalTok{      Mean   }\SpecialCharTok{:}\FloatTok{2.341}\NormalTok{   Mean   }\SpecialCharTok{:}\FloatTok{2.455}    
    \DecValTok{3}\NormalTok{rd Qu.}\SpecialCharTok{:}\FloatTok{40.00}   \DecValTok{3}\NormalTok{rd Qu.}\SpecialCharTok{:}\FloatTok{5.000}      \DecValTok{3}\NormalTok{rd Qu.}\SpecialCharTok{:}\FloatTok{3.000}   \DecValTok{3}\NormalTok{rd Qu.}\SpecialCharTok{:}\FloatTok{3.000}    
\NormalTok{    Max.   }\SpecialCharTok{:}\FloatTok{56.00}\NormalTok{   Max.   }\SpecialCharTok{:}\FloatTok{6.000}\NormalTok{      Max.   }\SpecialCharTok{:}\FloatTok{6.000}\NormalTok{   Max.   }\SpecialCharTok{:}\FloatTok{6.000}    
                                                                        
\NormalTok{     credit.limit   revolving.balance available.credit transaction.amount}\FloatTok{.12}
\NormalTok{    Min.   }\SpecialCharTok{:} \DecValTok{1438}\NormalTok{   Min.   }\SpecialCharTok{:}   \DecValTok{0}\NormalTok{      Min.   }\SpecialCharTok{:}    \DecValTok{3}\NormalTok{    Min.   }\SpecialCharTok{:}  \DecValTok{510}        
    \DecValTok{1}\NormalTok{st Qu.}\SpecialCharTok{:} \DecValTok{2555}   \DecValTok{1}\NormalTok{st Qu.}\SpecialCharTok{:} \DecValTok{359}      \DecValTok{1}\NormalTok{st Qu.}\SpecialCharTok{:} \DecValTok{1324}    \DecValTok{1}\NormalTok{st Qu.}\SpecialCharTok{:} \DecValTok{2156}        
\NormalTok{    Median }\SpecialCharTok{:} \DecValTok{4549}\NormalTok{   Median }\SpecialCharTok{:}\DecValTok{1276}\NormalTok{      Median }\SpecialCharTok{:} \DecValTok{3474}\NormalTok{    Median }\SpecialCharTok{:} \DecValTok{3899}        
\NormalTok{    Mean   }\SpecialCharTok{:} \DecValTok{8632}\NormalTok{   Mean   }\SpecialCharTok{:}\DecValTok{1163}\NormalTok{      Mean   }\SpecialCharTok{:} \DecValTok{7469}\NormalTok{    Mean   }\SpecialCharTok{:} \DecValTok{4404}        
    \DecValTok{3}\NormalTok{rd Qu.}\SpecialCharTok{:}\DecValTok{11068}   \DecValTok{3}\NormalTok{rd Qu.}\SpecialCharTok{:}\DecValTok{1784}      \DecValTok{3}\NormalTok{rd Qu.}\SpecialCharTok{:} \DecValTok{9859}    \DecValTok{3}\NormalTok{rd Qu.}\SpecialCharTok{:} \DecValTok{4741}        
\NormalTok{    Max.   }\SpecialCharTok{:}\DecValTok{34516}\NormalTok{   Max.   }\SpecialCharTok{:}\DecValTok{2517}\NormalTok{      Max.   }\SpecialCharTok{:}\DecValTok{34516}\NormalTok{    Max.   }\SpecialCharTok{:}\DecValTok{18484}        
                                                                            
\NormalTok{    transaction.count}\FloatTok{.12}\NormalTok{ ratio.amount.Q4.Q1 ratio.count.Q4.Q1 utilization.ratio}
\NormalTok{    Min.   }\SpecialCharTok{:} \FloatTok{10.00}\NormalTok{       Min.   }\SpecialCharTok{:}\FloatTok{0.0000}\NormalTok{     Min.   }\SpecialCharTok{:}\FloatTok{0.0000}\NormalTok{    Min.   }\SpecialCharTok{:}\FloatTok{0.0000}   
    \DecValTok{1}\NormalTok{st Qu.}\SpecialCharTok{:} \FloatTok{45.00}       \DecValTok{1}\NormalTok{st Qu.}\SpecialCharTok{:}\FloatTok{0.6310}     \DecValTok{1}\NormalTok{st Qu.}\SpecialCharTok{:}\FloatTok{0.5820}    \DecValTok{1}\NormalTok{st Qu.}\SpecialCharTok{:}\FloatTok{0.0230}   
\NormalTok{    Median }\SpecialCharTok{:} \FloatTok{67.00}\NormalTok{       Median }\SpecialCharTok{:}\FloatTok{0.7360}\NormalTok{     Median }\SpecialCharTok{:}\FloatTok{0.7020}\NormalTok{    Median }\SpecialCharTok{:}\FloatTok{0.1760}   
\NormalTok{    Mean   }\SpecialCharTok{:} \FloatTok{64.86}\NormalTok{       Mean   }\SpecialCharTok{:}\FloatTok{0.7599}\NormalTok{     Mean   }\SpecialCharTok{:}\FloatTok{0.7122}\NormalTok{    Mean   }\SpecialCharTok{:}\FloatTok{0.2749}   
    \DecValTok{3}\NormalTok{rd Qu.}\SpecialCharTok{:} \FloatTok{81.00}       \DecValTok{3}\NormalTok{rd Qu.}\SpecialCharTok{:}\FloatTok{0.8590}     \DecValTok{3}\NormalTok{rd Qu.}\SpecialCharTok{:}\FloatTok{0.8180}    \DecValTok{3}\NormalTok{rd Qu.}\SpecialCharTok{:}\FloatTok{0.5030}   
\NormalTok{    Max.   }\SpecialCharTok{:}\FloatTok{139.00}\NormalTok{       Max.   }\SpecialCharTok{:}\FloatTok{3.3970}\NormalTok{     Max.   }\SpecialCharTok{:}\FloatTok{3.7140}\NormalTok{    Max.   }\SpecialCharTok{:}\FloatTok{0.9990}   
                                                                               
\NormalTok{    churn     }
\NormalTok{    yes}\SpecialCharTok{:}\DecValTok{1627}  
\NormalTok{    no }\SpecialCharTok{:}\DecValTok{8500}  
              
              
              
              
   
\end{Highlighting}
\end{Shaded}

The summary statistics provide a first quantitative overview of the
dataset. While most variables fall within reasonable ranges, several
broad patterns emerge:

\begin{itemize}
\item
  \emph{Demographics and tenure:} Customers are primarily middle-aged
  (mean age around 46 years) and have maintained their accounts for an
  average of three years. Most fall within the 41--52 age range,
  suggesting a stable working-age customer base.
\item
  \emph{Credit behavior:} The average credit limit is about \$8,600 but
  varies widely, indicating substantial heterogeneity in financial
  profiles. Available credit closely mirrors the credit limit, while
  utilization ratios range from near zero to almost full usage,
  reflecting distinct groups of conservative and high-usage customers.
\item
  \emph{Transaction activity:} On average, customers complete about 65
  transactions per year with a total value near \$4,400. The presence of
  high-value spenders in the upper quartile points to diverse spending
  patterns that may influence churn behavior.
\item
  \emph{Behavioral changes:} Quarterly spending ratios show that
  customers generally spend slightly less toward the end of the year,
  though some exhibit sharp increases. These shifts may signal changing
  engagement or seasonal effects.
\item
  \emph{Categorical variables:} Females make up a slight majority, and
  most customers are married. Education levels are concentrated in the
  graduate and college categories, while income skews toward lower
  brackets. Nearly all customers hold blue credit cards, showing limited
  diversity in card types.
\end{itemize}

These descriptive patterns highlight the heterogeneity of the customer
base and suggest that scaling and transformation may be necessary for
several numerical variables. Some categorical features (particularly
\texttt{education}, \texttt{marital}, and \texttt{income}) contain an
\texttt{"unknown"} category, representing missing information that must
be handled carefully.

The next subsection focuses on preparing the \emph{churnCredit} dataset
for exploration by addressing missing values, verifying variable types,
and ensuring consistent formats. Proper preparation at this stage
ensures that the insights drawn from exploratory data analysis are both
valid and interpretable.

\subsection*{\texorpdfstring{Data Preparation for the \emph{churnCredit}
Dataset}{Data Preparation for the churnCredit Dataset}}\label{data-preparation-for-the-churncredit-dataset}
\addcontentsline{toc}{subsection}{Data Preparation for the
\emph{churnCredit} Dataset}

The initial inspection of the \emph{churnCredit} dataset revealed
several data quality issues that must be addressed before proceeding
with Exploratory Data Analysis (EDA). In particular, several categorical
variables (\texttt{education}, \texttt{income}, and \texttt{marital})
contain missing entries that were encoded as \texttt{"unknown"}.
Treating these values appropriately is essential to prevent biased
insights and ensure that the subsequent analyses reflect the true
characteristics of the data.

To standardize the representation of missing values, all
\texttt{"unknown"} entries are replaced with \texttt{NA}. The
\texttt{droplevels()} function is then applied to remove unused factor
levels:

\begin{Shaded}
\begin{Highlighting}[]
\NormalTok{churnCredit[churnCredit }\SpecialCharTok{==} \StringTok{"unknown"}\NormalTok{] }\OtherTok{\textless{}{-}} \ConstantTok{NA}
\NormalTok{churnCredit }\OtherTok{\textless{}{-}} \FunctionTok{droplevels}\NormalTok{(churnCredit)}
\end{Highlighting}
\end{Shaded}

Before imputing missing values, it is useful to visualize their extent
and distribution across variables. The \textbf{naniar} package provides
intuitive tools for this purpose. The \texttt{gg\_miss\_var()} function
displays the proportion of missing observations for each variable:

\begin{Shaded}
\begin{Highlighting}[]
\FunctionTok{library}\NormalTok{(naniar)}

\FunctionTok{gg\_miss\_var}\NormalTok{(churnCredit, }\AttributeTok{show\_pct =} \ConstantTok{TRUE}\NormalTok{)}
\end{Highlighting}
\end{Shaded}

\begin{center}
\includegraphics[width=0.65\linewidth,height=\textheight,keepaspectratio]{4-Exploratory-data-analysis_files/figure-pdf/unnamed-chunk-7-1.pdf}
\end{center}

The resulting plot shows that three categorical variables
(\texttt{education}, \texttt{income}, and \texttt{marital}) contain
missing values, with \texttt{education} having the highest proportion
(approximately 15\%). Although the level of missingness is modest,
addressing it is important to maintain data integrity and comparability
across groups.

There are several strategies for handling missing categorical data,
including mode imputation, random assignment, or introducing a separate
``missing'' category. Mode imputation would overrepresent the most
common category and distort group comparisons. Random imputation
preserves the original distribution. Thus, for this dataset, we apply
random imputation using the \textbf{Hmisc} function \texttt{impute()},
which replaces each missing value with a randomly sampled observed value
from the same variable. This method preserves the original distribution
of categories and avoids overrepresentation of the most frequent class.

\begin{Shaded}
\begin{Highlighting}[]
\FunctionTok{library}\NormalTok{(Hmisc)}

\NormalTok{churnCredit}\SpecialCharTok{$}\NormalTok{education }\OtherTok{\textless{}{-}} \FunctionTok{impute}\NormalTok{(churnCredit}\SpecialCharTok{$}\NormalTok{education, }\StringTok{"random"}\NormalTok{)}
\NormalTok{churnCredit}\SpecialCharTok{$}\NormalTok{income    }\OtherTok{\textless{}{-}} \FunctionTok{impute}\NormalTok{(churnCredit}\SpecialCharTok{$}\NormalTok{income, }\StringTok{"random"}\NormalTok{)}
\NormalTok{churnCredit}\SpecialCharTok{$}\NormalTok{marital   }\OtherTok{\textless{}{-}} \FunctionTok{impute}\NormalTok{(churnCredit}\SpecialCharTok{$}\NormalTok{marital, }\StringTok{"random"}\NormalTok{)}
\end{Highlighting}
\end{Shaded}

After imputing missing values, it is good practice to verify that all
variables are correctly typed. Categorical variables should be stored as
factors, while numerical features should be numeric. This ensures that
statistical summaries and visualizations behave as expected:

\begin{Shaded}
\begin{Highlighting}[]
\FunctionTok{str}\NormalTok{(churnCredit)}
   \StringTok{\textquotesingle{}data.frame\textquotesingle{}}\SpecialCharTok{:}    \DecValTok{10127}\NormalTok{ obs. of  }\DecValTok{21}\NormalTok{ variables}\SpecialCharTok{:}
    \ErrorTok{$}\NormalTok{ customer.ID          }\SpecialCharTok{:}\NormalTok{ int  }\DecValTok{768805383} \DecValTok{818770008} \DecValTok{713982108} \DecValTok{769911858} \DecValTok{709106358} \DecValTok{713061558} \DecValTok{810347208} \DecValTok{818906208} \DecValTok{710930508} \DecValTok{719661558}\NormalTok{ ...}
    \SpecialCharTok{$}\NormalTok{ age                  }\SpecialCharTok{:}\NormalTok{ int  }\DecValTok{45} \DecValTok{49} \DecValTok{51} \DecValTok{40} \DecValTok{40} \DecValTok{44} \DecValTok{51} \DecValTok{32} \DecValTok{37} \DecValTok{48}\NormalTok{ ...}
    \SpecialCharTok{$}\NormalTok{ gender               }\SpecialCharTok{:}\NormalTok{ Factor w}\SpecialCharTok{/} \DecValTok{2}\NormalTok{ levels }\StringTok{"female"}\NormalTok{,}\StringTok{"male"}\SpecialCharTok{:} \DecValTok{2} \DecValTok{1} \DecValTok{2} \DecValTok{1} \DecValTok{2} \DecValTok{2} \DecValTok{2} \DecValTok{2} \DecValTok{2} \DecValTok{2}\NormalTok{ ...}
    \SpecialCharTok{$}\NormalTok{ education            }\SpecialCharTok{:}\NormalTok{ Factor w}\SpecialCharTok{/} \DecValTok{6}\NormalTok{ levels }\StringTok{"uneducated"}\NormalTok{,}\StringTok{"highschool"}\NormalTok{,..}\SpecialCharTok{:} \DecValTok{2} \DecValTok{4} \DecValTok{4} \DecValTok{2} \DecValTok{1} \DecValTok{4} \DecValTok{4} \DecValTok{2} \DecValTok{1} \DecValTok{4}\NormalTok{ ...}
\NormalTok{     ..}\SpecialCharTok{{-}} \FunctionTok{attr}\NormalTok{(}\SpecialCharTok{*}\NormalTok{, }\StringTok{"imputed"}\NormalTok{)}\OtherTok{=}\NormalTok{ int [}\DecValTok{1}\SpecialCharTok{:}\DecValTok{1519}\NormalTok{] }\DecValTok{7} \DecValTok{12} \DecValTok{16} \DecValTok{18} \DecValTok{24} \DecValTok{25} \DecValTok{28} \DecValTok{31} \DecValTok{42} \DecValTok{51}\NormalTok{ ...}
    \SpecialCharTok{$}\NormalTok{ marital              }\SpecialCharTok{:}\NormalTok{ Factor w}\SpecialCharTok{/} \DecValTok{3}\NormalTok{ levels }\StringTok{"married"}\NormalTok{,}\StringTok{"single"}\NormalTok{,..}\SpecialCharTok{:} \DecValTok{1} \DecValTok{2} \DecValTok{1} \DecValTok{2} \DecValTok{1} \DecValTok{1} \DecValTok{1} \DecValTok{1} \DecValTok{2} \DecValTok{2}\NormalTok{ ...}
\NormalTok{     ..}\SpecialCharTok{{-}} \FunctionTok{attr}\NormalTok{(}\SpecialCharTok{*}\NormalTok{, }\StringTok{"imputed"}\NormalTok{)}\OtherTok{=}\NormalTok{ int [}\DecValTok{1}\SpecialCharTok{:}\DecValTok{749}\NormalTok{] }\DecValTok{4} \DecValTok{8} \DecValTok{11} \DecValTok{14} \DecValTok{16} \DecValTok{27} \DecValTok{39} \DecValTok{56} \DecValTok{73} \DecValTok{82}\NormalTok{ ...}
    \SpecialCharTok{$}\NormalTok{ income               }\SpecialCharTok{:}\NormalTok{ Factor w}\SpecialCharTok{/} \DecValTok{5}\NormalTok{ levels }\StringTok{"\textless{}40K"}\NormalTok{,}\StringTok{"40K{-}60K"}\NormalTok{,..}\SpecialCharTok{:} \DecValTok{3} \DecValTok{1} \DecValTok{4} \DecValTok{1} \DecValTok{3} \DecValTok{2} \DecValTok{5} \DecValTok{3} \DecValTok{3} \DecValTok{4}\NormalTok{ ...}
\NormalTok{     ..}\SpecialCharTok{{-}} \FunctionTok{attr}\NormalTok{(}\SpecialCharTok{*}\NormalTok{, }\StringTok{"imputed"}\NormalTok{)}\OtherTok{=}\NormalTok{ int [}\DecValTok{1}\SpecialCharTok{:}\DecValTok{1112}\NormalTok{] }\DecValTok{20} \DecValTok{29} \DecValTok{40} \DecValTok{45} \DecValTok{59} \DecValTok{84} \DecValTok{95} \DecValTok{101} \DecValTok{102} \DecValTok{139}\NormalTok{ ...}
    \SpecialCharTok{$}\NormalTok{ card.category        }\SpecialCharTok{:}\NormalTok{ Factor w}\SpecialCharTok{/} \DecValTok{4}\NormalTok{ levels }\StringTok{"blue"}\NormalTok{,}\StringTok{"silver"}\NormalTok{,..}\SpecialCharTok{:} \DecValTok{1} \DecValTok{1} \DecValTok{1} \DecValTok{1} \DecValTok{1} \DecValTok{1} \DecValTok{3} \DecValTok{2} \DecValTok{1} \DecValTok{1}\NormalTok{ ...}
    \SpecialCharTok{$}\NormalTok{ dependent.count      }\SpecialCharTok{:}\NormalTok{ int  }\DecValTok{3} \DecValTok{5} \DecValTok{3} \DecValTok{4} \DecValTok{3} \DecValTok{2} \DecValTok{4} \DecValTok{0} \DecValTok{3} \DecValTok{2}\NormalTok{ ...}
    \SpecialCharTok{$}\NormalTok{ months.on.book       }\SpecialCharTok{:}\NormalTok{ int  }\DecValTok{39} \DecValTok{44} \DecValTok{36} \DecValTok{34} \DecValTok{21} \DecValTok{36} \DecValTok{46} \DecValTok{27} \DecValTok{36} \DecValTok{36}\NormalTok{ ...}
    \SpecialCharTok{$}\NormalTok{ relationship.count   }\SpecialCharTok{:}\NormalTok{ int  }\DecValTok{5} \DecValTok{6} \DecValTok{4} \DecValTok{3} \DecValTok{5} \DecValTok{3} \DecValTok{6} \DecValTok{2} \DecValTok{5} \DecValTok{6}\NormalTok{ ...}
    \SpecialCharTok{$}\NormalTok{ months.inactive      }\SpecialCharTok{:}\NormalTok{ int  }\DecValTok{1} \DecValTok{1} \DecValTok{1} \DecValTok{4} \DecValTok{1} \DecValTok{1} \DecValTok{1} \DecValTok{2} \DecValTok{2} \DecValTok{3}\NormalTok{ ...}
    \SpecialCharTok{$}\NormalTok{ contacts.count}\FloatTok{.12}    \SpecialCharTok{:}\NormalTok{ int  }\DecValTok{3} \DecValTok{2} \DecValTok{0} \DecValTok{1} \DecValTok{0} \DecValTok{2} \DecValTok{3} \DecValTok{2} \DecValTok{0} \DecValTok{3}\NormalTok{ ...}
    \SpecialCharTok{$}\NormalTok{ credit.limit         }\SpecialCharTok{:}\NormalTok{ num  }\DecValTok{12691} \DecValTok{8256} \DecValTok{3418} \DecValTok{3313} \DecValTok{4716}\NormalTok{ ...}
    \SpecialCharTok{$}\NormalTok{ revolving.balance    }\SpecialCharTok{:}\NormalTok{ int  }\DecValTok{777} \DecValTok{864} \DecValTok{0} \DecValTok{2517} \DecValTok{0} \DecValTok{1247} \DecValTok{2264} \DecValTok{1396} \DecValTok{2517} \DecValTok{1677}\NormalTok{ ...}
    \SpecialCharTok{$}\NormalTok{ available.credit     }\SpecialCharTok{:}\NormalTok{ num  }\DecValTok{11914} \DecValTok{7392} \DecValTok{3418} \DecValTok{796} \DecValTok{4716}\NormalTok{ ...}
    \SpecialCharTok{$}\NormalTok{ transaction.amount}\FloatTok{.12}\SpecialCharTok{:}\NormalTok{ int  }\DecValTok{1144} \DecValTok{1291} \DecValTok{1887} \DecValTok{1171} \DecValTok{816} \DecValTok{1088} \DecValTok{1330} \DecValTok{1538} \DecValTok{1350} \DecValTok{1441}\NormalTok{ ...}
    \SpecialCharTok{$}\NormalTok{ transaction.count}\FloatTok{.12} \SpecialCharTok{:}\NormalTok{ int  }\DecValTok{42} \DecValTok{33} \DecValTok{20} \DecValTok{20} \DecValTok{28} \DecValTok{24} \DecValTok{31} \DecValTok{36} \DecValTok{24} \DecValTok{32}\NormalTok{ ...}
    \SpecialCharTok{$}\NormalTok{ ratio.amount.Q4.Q1   }\SpecialCharTok{:}\NormalTok{ num  }\FloatTok{1.33} \FloatTok{1.54} \FloatTok{2.59} \FloatTok{1.41} \FloatTok{2.17}\NormalTok{ ...}
    \SpecialCharTok{$}\NormalTok{ ratio.count.Q4.Q1    }\SpecialCharTok{:}\NormalTok{ num  }\FloatTok{1.62} \FloatTok{3.71} \FloatTok{2.33} \FloatTok{2.33} \FloatTok{2.5}\NormalTok{ ...}
    \SpecialCharTok{$}\NormalTok{ utilization.ratio    }\SpecialCharTok{:}\NormalTok{ num  }\FloatTok{0.061} \FloatTok{0.105} \DecValTok{0} \FloatTok{0.76} \DecValTok{0} \FloatTok{0.311} \FloatTok{0.066} \FloatTok{0.048} \FloatTok{0.113} \FloatTok{0.144}\NormalTok{ ...}
    \SpecialCharTok{$}\NormalTok{ churn                }\SpecialCharTok{:}\NormalTok{ Factor w}\SpecialCharTok{/} \DecValTok{2}\NormalTok{ levels }\StringTok{"yes"}\NormalTok{,}\StringTok{"no"}\SpecialCharTok{:} \DecValTok{2} \DecValTok{2} \DecValTok{2} \DecValTok{2} \DecValTok{2} \DecValTok{2} \DecValTok{2} \DecValTok{2} \DecValTok{2} \DecValTok{2}\NormalTok{ ...}
\end{Highlighting}
\end{Shaded}

Once the data are clean and variables properly defined, the dataset is
ready for exploratory analysis. Throughout this section, we use the
cleaned version of the dataset. The next section applies summary
statistics and visualization techniques to uncover key patterns and
relationships underlying customer churn.

\section{Exploring Categorical Features}\label{sec-EDA-categorical}

Categorical variables group observations into distinct classes that
often reflect demographic or behavioural characteristics. In the
\emph{churnCredit} dataset, the key categorical features include
\texttt{gender}, \texttt{education}, \texttt{marital},
\texttt{card.category}, and \texttt{churn}. Examining how these
variables are distributed, and how they relate to the outcome
\texttt{churn}, provides early clues about customer loyalty and
disengagement.

We begin with the distribution of the target variable \texttt{churn},
which indicates whether a customer has closed a credit card account.
Understanding this distribution is essential for assessing class
balance, a factor that directly influences model training and
interpretation. The bar plot and pie chart below summarise the
proportion of customers who churned:

\begin{Shaded}
\begin{Highlighting}[]
\FunctionTok{library}\NormalTok{(ggplot2)}

\CommentTok{\# Bar plot}
\FunctionTok{ggplot}\NormalTok{(}\AttributeTok{data =}\NormalTok{ churnCredit, }\FunctionTok{aes}\NormalTok{(}\AttributeTok{x =}\NormalTok{ churn, }\AttributeTok{label =}\NormalTok{ scales}\SpecialCharTok{::}\FunctionTok{percent}\NormalTok{(}\FunctionTok{prop.table}\NormalTok{(}\FunctionTok{after\_stat}\NormalTok{(count))))) }\SpecialCharTok{+}
  \FunctionTok{geom\_bar}\NormalTok{(}\AttributeTok{fill =} \FunctionTok{c}\NormalTok{(}\StringTok{"\#F4A582"}\NormalTok{, }\StringTok{"\#A8D5BA"}\NormalTok{)) }\SpecialCharTok{+}
  \FunctionTok{geom\_text}\NormalTok{(}\AttributeTok{stat =} \StringTok{"count"}\NormalTok{, }\AttributeTok{vjust =} \FloatTok{0.4}\NormalTok{, }\AttributeTok{size =} \DecValTok{6}\NormalTok{)}

\CommentTok{\# Pie chart}
\FunctionTok{ggplot}\NormalTok{(churnCredit, }\FunctionTok{aes}\NormalTok{(}\AttributeTok{x =} \StringTok{""}\NormalTok{, }\AttributeTok{fill =}\NormalTok{ churn)) }\SpecialCharTok{+}
  \FunctionTok{geom\_bar}\NormalTok{(}\AttributeTok{width =} \DecValTok{1}\NormalTok{) }\SpecialCharTok{+}
  \FunctionTok{coord\_polar}\NormalTok{(}\AttributeTok{theta =} \StringTok{"y"}\NormalTok{) }\SpecialCharTok{+}
  \FunctionTok{theme\_void}\NormalTok{()}
\end{Highlighting}
\end{Shaded}

\begin{figure}

\begin{minipage}{0.50\linewidth}
\begin{center}
\pandocbounded{\includegraphics[keepaspectratio]{4-Exploratory-data-analysis_files/figure-pdf/unnamed-chunk-10-1.pdf}}
\end{center}
\end{minipage}%
%
\begin{minipage}{0.50\linewidth}
\begin{center}
\pandocbounded{\includegraphics[keepaspectratio]{4-Exploratory-data-analysis_files/figure-pdf/unnamed-chunk-10-2.pdf}}
\end{center}
\end{minipage}%

\end{figure}%

The pie-chart code is slightly more involved because it uses polar
coordinates. Do not worry if it feels unfamiliar: most visualisations in
this book rely on simpler bar-plot structures that are easier to read
and adapt. The left panel presents the bar plot, while the right panel
offers a compact view of the same proportions using a pie chart. Both
highlight that most customers remain active (\texttt{churn\ =\ "no"}),
with only a small proportion (about 16.1 percent) closing their
accounts. This imbalance reflects patterns commonly seen in financial
and subscription-based datasets.

Although pie charts are less useful when comparing multiple groups, they
can be effective for displaying a single binary variable. A basic
version of the bar plot, without colours or percentage labels, can be
created with:

\begin{Shaded}
\begin{Highlighting}[]
\FunctionTok{ggplot}\NormalTok{(}\AttributeTok{data =}\NormalTok{ churnCredit) }\SpecialCharTok{+}
  \FunctionTok{geom\_bar}\NormalTok{(}\FunctionTok{aes}\NormalTok{(}\AttributeTok{x =}\NormalTok{ churn))}
\end{Highlighting}
\end{Shaded}

The simpler display offers a quick overview, whereas the enhanced
version communicates proportions more clearly. Such refinements improve
interpretability, especially when results are presented to non-technical
stakeholders.

\begin{quote}
\emph{Try if yourself:} Create a simple bar plot of the \texttt{gender}
feature using \textbf{ggplot2}. You may also experiment with adding
colour fills or percentage labels. This short activity reinforces the
basic structure of bar plots before we examine categorical relationships
in more detail.
\end{quote}

Class imbalance is more than a descriptive observation: it can
substantially influence modelling performance. Algorithms trained on
imbalanced data often favour the majority class, which leads to biased
predictions and reduced sensitivity to minority outcomes. Strategies for
addressing this issue are introduced in Chapter
\textbf{?@sec-ch6-setup-data}, Section \textbf{?@sec-ch6-balancing}.

Having established the distribution of the target variable, we next
examine how other categorical features relate to churn. These
relationships may reveal customer segments or behavioural patterns
associated with increased attrition risk.

\subsection*{Relationship Between Gender and
Churn}\label{relationship-between-gender-and-churn}
\addcontentsline{toc}{subsection}{Relationship Between Gender and Churn}

Among the demographic variables, \texttt{gender} can help examine
whether customer retention behaviour differs between male and female
account holders. Although gender is not typically a strong predictor of
churn in financial services, even small differences may offer useful
insights into customer engagement patterns.

\begin{Shaded}
\begin{Highlighting}[]
\FunctionTok{ggplot}\NormalTok{(}\AttributeTok{data =}\NormalTok{ churnCredit) }\SpecialCharTok{+} 
  \FunctionTok{geom\_bar}\NormalTok{(}\FunctionTok{aes}\NormalTok{(}\AttributeTok{x =}\NormalTok{ gender, }\AttributeTok{fill =}\NormalTok{ churn))    }

\FunctionTok{ggplot}\NormalTok{(}\AttributeTok{data =}\NormalTok{ churnCredit) }\SpecialCharTok{+} 
  \FunctionTok{geom\_bar}\NormalTok{(}\FunctionTok{aes}\NormalTok{(}\AttributeTok{x =}\NormalTok{ gender, }\AttributeTok{fill =}\NormalTok{ churn), }\AttributeTok{position =} \StringTok{"fill"}\NormalTok{) }
\end{Highlighting}
\end{Shaded}

\begin{figure}

\begin{minipage}{0.50\linewidth}
\begin{center}
\pandocbounded{\includegraphics[keepaspectratio]{4-Exploratory-data-analysis_files/figure-pdf/unnamed-chunk-11-1.pdf}}
\end{center}
\end{minipage}%
%
\begin{minipage}{0.50\linewidth}
\begin{center}
\pandocbounded{\includegraphics[keepaspectratio]{4-Exploratory-data-analysis_files/figure-pdf/unnamed-chunk-11-2.pdf}}
\end{center}
\end{minipage}%

\end{figure}%

The left plot presents the absolute numbers of churners and non-churners
by gender. The right plot, which shows proportions within each gender
group, highlights relative differences in churn rates. Both displays
suggest that the churn rate is slightly higher among female customers.
However, the difference is modest, indicating that gender alone is not a
major factor influencing account closure.

To examine the pattern more closely, we can inspect the contingency
table:

\begin{Shaded}
\begin{Highlighting}[]
\FunctionTok{addmargins}\NormalTok{(}\FunctionTok{table}\NormalTok{(churnCredit}\SpecialCharTok{$}\NormalTok{churn, churnCredit}\SpecialCharTok{$}\NormalTok{gender,}
                 \AttributeTok{dnn =} \FunctionTok{c}\NormalTok{(}\StringTok{"Churn"}\NormalTok{, }\StringTok{"Gender"}\NormalTok{)))}
\NormalTok{        Gender}
\NormalTok{   Churn female  male   Sum}
\NormalTok{     yes    }\DecValTok{930}   \DecValTok{697}  \DecValTok{1627}
\NormalTok{     no    }\DecValTok{4428}  \DecValTok{4072}  \DecValTok{8500}
\NormalTok{     Sum   }\DecValTok{5358}  \DecValTok{4769} \DecValTok{10127}
\end{Highlighting}
\end{Shaded}

The table confirms the visual impression: the proportion of female
customers who churn is marginally higher than that of male customers.
This small difference may reflect minor behavioural or engagement
variations rather than any structural or policy-related factor.

From an analytical perspective, this suggests that gender is not a key
differentiating variable for churn behaviour. More substantial variation
is likely to emerge from behavioural and financial indicators such as
transaction activity, credit utilisation, and customer service
interactions, which generally offer stronger predictive value.

\begin{quote}
\emph{Try it yourself:} Compute the churn rate separately for male and
female customers using the \emph{churnCredit} dataset. Create your own
bar plot and compare it with the figure shown above. Based on the
observed proportions, would you expect the difference in churn rates to
be statistically significant? We revisit this question formally in the
next Chapter (Section \textbf{?@sec-ch5-two-sample-z-test}) when
introducing the test for two proportions.
\end{quote}

\subsection*{Relationship Between Card Category and
Churn}\label{relationship-between-card-category-and-churn}
\addcontentsline{toc}{subsection}{Relationship Between Card Category and
Churn}

Credit card type is one of the most informative service features in the
\emph{churnCredit} dataset. The variable \texttt{card.category}
distinguishes four tiers---blue, silver, gold, and
platinum---representing different benefit levels and customer segments.

\begin{Shaded}
\begin{Highlighting}[]
\FunctionTok{ggplot}\NormalTok{(}\AttributeTok{data =}\NormalTok{ churnCredit) }\SpecialCharTok{+} 
  \FunctionTok{geom\_bar}\NormalTok{(}\FunctionTok{aes}\NormalTok{(}\AttributeTok{x =}\NormalTok{ card.category, }\AttributeTok{fill =}\NormalTok{ churn)) }\SpecialCharTok{+} 
  \FunctionTok{labs}\NormalTok{(}\AttributeTok{x =} \StringTok{"Card Category"}\NormalTok{, }\AttributeTok{y =} \StringTok{"Count"}\NormalTok{)}

\FunctionTok{ggplot}\NormalTok{(}\AttributeTok{data =}\NormalTok{ churnCredit) }\SpecialCharTok{+} 
  \FunctionTok{geom\_bar}\NormalTok{(}\FunctionTok{aes}\NormalTok{(}\AttributeTok{x =}\NormalTok{ card.category, }\AttributeTok{fill =}\NormalTok{ churn), }\AttributeTok{position =} \StringTok{"fill"}\NormalTok{) }\SpecialCharTok{+} 
  \FunctionTok{labs}\NormalTok{(}\AttributeTok{x =} \StringTok{"Card Category"}\NormalTok{, }\AttributeTok{y =} \StringTok{"Proportion"}\NormalTok{)}
\end{Highlighting}
\end{Shaded}

\begin{figure}

\begin{minipage}{0.50\linewidth}
\begin{center}
\pandocbounded{\includegraphics[keepaspectratio]{4-Exploratory-data-analysis_files/figure-pdf/unnamed-chunk-13-1.pdf}}
\end{center}
\end{minipage}%
%
\begin{minipage}{0.50\linewidth}
\begin{center}
\pandocbounded{\includegraphics[keepaspectratio]{4-Exploratory-data-analysis_files/figure-pdf/unnamed-chunk-13-2.pdf}}
\end{center}
\end{minipage}%

\end{figure}%

The left plot shows the absolute number of churners and non-churners
within each tier, while the right plot presents proportions within card
categories. The distribution is highly imbalanced: more than 93 percent
of customers hold a blue card, which is typically the entry-level or
no-fee version. This reflects real-world product portfolios, where most
clients hold standard cards. Because the remaining groups are small,
differences across tiers must be interpreted cautiously.

\begin{Shaded}
\begin{Highlighting}[]
\FunctionTok{addmargins}\NormalTok{(}\FunctionTok{table}\NormalTok{(churnCredit}\SpecialCharTok{$}\NormalTok{churn, churnCredit}\SpecialCharTok{$}\NormalTok{card.category, }
                 \AttributeTok{dnn =} \FunctionTok{c}\NormalTok{(}\StringTok{"Churn"}\NormalTok{, }\StringTok{"Card Category"}\NormalTok{)))}
\NormalTok{        Card Category}
\NormalTok{   Churn  blue silver  gold platinum   Sum}
\NormalTok{     yes  }\DecValTok{1519}     \DecValTok{82}    \DecValTok{21}        \DecValTok{5}  \DecValTok{1627}
\NormalTok{     no   }\DecValTok{7917}    \DecValTok{473}    \DecValTok{95}       \DecValTok{15}  \DecValTok{8500}
\NormalTok{     Sum  }\DecValTok{9436}    \DecValTok{555}   \DecValTok{116}       \DecValTok{20} \DecValTok{10127}
\end{Highlighting}
\end{Shaded}

The contingency table confirms that churn rates are somewhat higher
among holders of blue and silver cards and lower among customers with
gold or platinum cards. These differences are modest, but they suggest
that premium cardholders tend to be more engaged and therefore less
likely to churn.

Because the silver, gold, and platinum categories are relatively small,
analysts often collapse similar groups to ensure sufficient sample sizes
for reliable modelling. A natural reclassification is to separate
``blue'' from ``silver+'' (a combined group of silver, gold, and
platinum cardholders). Such reclassification reduces sparsity,
stabilises estimates, and simplifies interpretation, especially in
downstream modelling.

\begin{quote}
\emph{Try it yourself:} Collapse the card categories into two groups,
``blue'' and ``silver+'', using the \texttt{fct\_collapse()} function
from the \textbf{forcats} package (as in Section
\textbf{?@sec-ch3-data-pre-adult}). Then recreate the bar plots to
compare churn patterns between the two groups. Does the collapsed
version make churn differences easier to interpret? Would this version
be more suitable for modelling?
\end{quote}

\subsection*{Relationship Between Income and
Churn}\label{relationship-between-income-and-churn}
\addcontentsline{toc}{subsection}{Relationship Between Income and Churn}

Income level reflects purchasing power and financial stability, both of
which may influence a customer's likelihood of closing a credit account.
The variable \texttt{income} in the \emph{churnCredit} dataset includes
five ordered categories, ranging from \emph{less than \$40K} to
\emph{over \$120K}. Since missing values were imputed earlier, the
variable now provides a complete and consistent basis for comparison.

\begin{Shaded}
\begin{Highlighting}[]
\FunctionTok{ggplot}\NormalTok{(}\AttributeTok{data =}\NormalTok{ churnCredit) }\SpecialCharTok{+} 
  \FunctionTok{geom\_bar}\NormalTok{(}\FunctionTok{aes}\NormalTok{(}\AttributeTok{x =}\NormalTok{ income, }\AttributeTok{fill =}\NormalTok{ churn)) }\SpecialCharTok{+} 
  \FunctionTok{labs}\NormalTok{(}\AttributeTok{x =} \StringTok{"Annual Income Bracket"}\NormalTok{, }\AttributeTok{y =} \StringTok{"Count"}\NormalTok{) }\SpecialCharTok{+}
  \FunctionTok{theme}\NormalTok{(}\AttributeTok{axis.text.x =} \FunctionTok{element\_text}\NormalTok{(}\AttributeTok{angle =} \DecValTok{45}\NormalTok{, }\AttributeTok{hjust =} \DecValTok{1}\NormalTok{))}

\FunctionTok{ggplot}\NormalTok{(}\AttributeTok{data =}\NormalTok{ churnCredit) }\SpecialCharTok{+} 
  \FunctionTok{geom\_bar}\NormalTok{(}\FunctionTok{aes}\NormalTok{(}\AttributeTok{x =}\NormalTok{ income, }\AttributeTok{fill =}\NormalTok{ churn), }\AttributeTok{position =} \StringTok{"fill"}\NormalTok{) }\SpecialCharTok{+} 
  \FunctionTok{labs}\NormalTok{(}\AttributeTok{x =} \StringTok{"Annual Income Bracket"}\NormalTok{, }\AttributeTok{y =} \StringTok{"Proportion"}\NormalTok{) }\SpecialCharTok{+}
  \FunctionTok{theme}\NormalTok{(}\AttributeTok{axis.text.x =} \FunctionTok{element\_text}\NormalTok{(}\AttributeTok{angle =} \DecValTok{45}\NormalTok{, }\AttributeTok{hjust =} \DecValTok{1}\NormalTok{))}
\end{Highlighting}
\end{Shaded}

\begin{figure}

\begin{minipage}{0.50\linewidth}
\begin{center}
\pandocbounded{\includegraphics[keepaspectratio]{4-Exploratory-data-analysis_files/figure-pdf/unnamed-chunk-15-1.pdf}}
\end{center}
\end{minipage}%
%
\begin{minipage}{0.50\linewidth}
\begin{center}
\pandocbounded{\includegraphics[keepaspectratio]{4-Exploratory-data-analysis_files/figure-pdf/unnamed-chunk-15-2.pdf}}
\end{center}
\end{minipage}%

\end{figure}%

The bar plots show a gradual decline in churn as income increases.
Customers in the lowest income bracket (less than \texttt{\$40K}) churn
slightly more often than those in higher income groups, whereas
customers earning over \texttt{\$120K} exhibit the lowest churn rate.
Although the trend is modest, the direction suggests that financial
capacity is associated with account stability.

\begin{Shaded}
\begin{Highlighting}[]
\FunctionTok{addmargins}\NormalTok{(}\FunctionTok{table}\NormalTok{(churnCredit}\SpecialCharTok{$}\NormalTok{churn, churnCredit}\SpecialCharTok{$}\NormalTok{income, }
                 \AttributeTok{dnn =} \FunctionTok{c}\NormalTok{(}\StringTok{"Churn"}\NormalTok{, }\StringTok{"Income"}\NormalTok{)))}
\NormalTok{        Income}
\NormalTok{   Churn  }\SpecialCharTok{\textless{}}\DecValTok{40}\NormalTok{K }\DecValTok{40}\NormalTok{K}\DecValTok{{-}60}\NormalTok{K }\DecValTok{60}\NormalTok{K}\DecValTok{{-}80}\NormalTok{K }\DecValTok{80}\NormalTok{K}\DecValTok{{-}120}\NormalTok{K }\SpecialCharTok{\textgreater{}}\DecValTok{120}\NormalTok{K   Sum}
\NormalTok{     yes   }\DecValTok{677}     \DecValTok{310}     \DecValTok{227}      \DecValTok{271}   \DecValTok{142}  \DecValTok{1627}
\NormalTok{     no   }\DecValTok{3327}    \DecValTok{1705}    \DecValTok{1345}     \DecValTok{1453}   \DecValTok{670}  \DecValTok{8500}
\NormalTok{     Sum  }\DecValTok{4004}    \DecValTok{2015}    \DecValTok{1572}     \DecValTok{1724}   \DecValTok{812} \DecValTok{10127}
\end{Highlighting}
\end{Shaded}

The contingency table confirms this pattern, showing small but
consistent differences across income levels. Lower-income customers may
be more responsive to service fees or credit limits, while higher-income
customers tend to maintain longer and more stable relationships with the
bank.

From an analytical perspective, this variable provides a mild yet
interpretable signal of churn behavior. Its ordinal structure may prove
useful in modeling if treated appropriately. For instance, by encoding
it as an ordered factor to preserve its inherent ranking rather than as
a set of independent categories.

\subsection*{Relationship Between Marital Status and
Churn}\label{relationship-between-marital-status-and-churn}
\addcontentsline{toc}{subsection}{Relationship Between Marital Status
and Churn}

Marital status may influence financial behaviour and account management,
making it a useful demographic variable to examine in the context of
churn. The \texttt{marital} variable in the \emph{churnCredit} dataset
includes three main categories: \emph{married}, \emph{single}, and
\emph{divorced}. Differences across these groups may reflect household
financial habits, shared responsibilities, or lifestyle-related spending
patterns.

\begin{Shaded}
\begin{Highlighting}[]
\FunctionTok{ggplot}\NormalTok{(}\AttributeTok{data =}\NormalTok{ churnCredit) }\SpecialCharTok{+} 
  \FunctionTok{geom\_bar}\NormalTok{(}\FunctionTok{aes}\NormalTok{(}\AttributeTok{x =}\NormalTok{ marital, }\AttributeTok{fill =}\NormalTok{ churn)) }\SpecialCharTok{+} 
  \FunctionTok{labs}\NormalTok{(}\AttributeTok{x =} \StringTok{"Marital Status"}\NormalTok{, }\AttributeTok{y =} \StringTok{"Count"}\NormalTok{)}

\FunctionTok{ggplot}\NormalTok{(}\AttributeTok{data =}\NormalTok{ churnCredit) }\SpecialCharTok{+} 
  \FunctionTok{geom\_bar}\NormalTok{(}\FunctionTok{aes}\NormalTok{(}\AttributeTok{x =}\NormalTok{ marital, }\AttributeTok{fill =}\NormalTok{ churn), }\AttributeTok{position =} \StringTok{"fill"}\NormalTok{) }\SpecialCharTok{+} 
  \FunctionTok{labs}\NormalTok{(}\AttributeTok{x =} \StringTok{"Marital Status"}\NormalTok{, }\AttributeTok{y =} \StringTok{"Proportion"}\NormalTok{)}
\end{Highlighting}
\end{Shaded}

\begin{figure}

\begin{minipage}{0.50\linewidth}
\begin{center}
\pandocbounded{\includegraphics[keepaspectratio]{4-Exploratory-data-analysis_files/figure-pdf/unnamed-chunk-17-1.pdf}}
\end{center}
\end{minipage}%
%
\begin{minipage}{0.50\linewidth}
\begin{center}
\pandocbounded{\includegraphics[keepaspectratio]{4-Exploratory-data-analysis_files/figure-pdf/unnamed-chunk-17-2.pdf}}
\end{center}
\end{minipage}%

\end{figure}%

The bar plots show that single customers churn slightly more often than
married or divorced customers. The differences, however, are relatively
small and may not be practically significant.

\begin{Shaded}
\begin{Highlighting}[]
\FunctionTok{addmargins}\NormalTok{(}\FunctionTok{table}\NormalTok{(churnCredit}\SpecialCharTok{$}\NormalTok{churn, churnCredit}\SpecialCharTok{$}\NormalTok{marital, }
                 \AttributeTok{dnn =} \FunctionTok{c}\NormalTok{(}\StringTok{"Churn"}\NormalTok{, }\StringTok{"Marital Status"}\NormalTok{)))}
\NormalTok{        Marital Status}
\NormalTok{   Churn married single divorced   Sum}
\NormalTok{     yes     }\DecValTok{767}    \DecValTok{727}      \DecValTok{133}  \DecValTok{1627}
\NormalTok{     no     }\DecValTok{4277}   \DecValTok{3548}      \DecValTok{675}  \DecValTok{8500}
\NormalTok{     Sum    }\DecValTok{5044}   \DecValTok{4275}      \DecValTok{808} \DecValTok{10127}
\end{Highlighting}
\end{Shaded}

The contingency table confirms this observation: marital status appears
to have only a minor association with churn. Although these differences
may reflect small lifestyle- or household-related behavioural patterns,
the overall relationship remains weak.

From an analytical perspective, marital status provides limited
explanatory value for predicting churn. Subsequent analysis should
therefore prioritise behavioural and transactional variables---such as
spending activity, utilisation ratio, and frequency of customer
contact---which typically provide stronger indicators of churn risk.
Because both \texttt{marital} and \texttt{churn} are categorical
variables with multiple levels, we will revisit this question formally
using the Chi-square test in the next chapter (Section
\textbf{?@sec-ch5-chi-square-test}).

\begin{quote}
\emph{Try it yourself:} Explore whether education level is associated
with churn in the \emph{churnCredit} dataset. Follow a similar approach
as in the marital-status example: create bar plots (counts and
proportions) for \texttt{education} against \texttt{churn}, and examine
the contingency table. Since \texttt{education} includes multiple
levels, consider whether any differences appear meaningful in practice.
\end{quote}

\subsection*{Summary of Categorical
Findings}\label{summary-of-categorical-findings}
\addcontentsline{toc}{subsection}{Summary of Categorical Findings}

The exploratory analysis of categorical variables in the
\emph{churnCredit} dataset reveals several subtle patterns that help
contextualise customer churn behaviour.

\begin{itemize}
\item
  Gender: Female customers exhibit a slightly higher churn rate than
  males, but the difference is small and unlikely to hold strong
  predictive value.
\item
  Card category: Customers with higher-tier cards (gold and platinum)
  are less likely to churn, suggesting that premium benefits and deeper
  engagement contribute to stronger retention.
\item
  Income: Churn rates decrease gradually as income rises, indicating
  that greater financial capacity may support long-term account
  stability.
\item
  Marital status: Single customers churn somewhat more often than
  married or divorced customers, possibly reflecting differing financial
  priorities or household financial behaviours.
\item
  Education: Lower education levels correspond to marginally higher
  churn, though the overall differences remain small and of limited
  explanatory value.
\end{itemize}

Overall, demographic characteristics such as gender, education, and
marital status exhibit only weak associations with churn. In contrast,
service-related and financial features (particularly card type and
income) offer more interpretable and potentially useful signals. These
findings suggest that behavioural and transactional indicators are
likely to play a more decisive role in understanding and predicting
customer attrition.

The next section continues the exploratory analysis by turning to
numerical variables that capture spending activity, credit utilisation,
and overall customer engagement.

\section{Exploring Numerical Features}\label{sec-EDA-sec-numeric}

The \emph{churnCredit} dataset contains fourteen numerical variables
that describe different aspects of customer behavior, credit management,
and engagement with the bank. Analyzing these variables helps uncover
how customers differ in spending, activity level, financial capacity,
and behavioral changes over time---factors that are often related to
churn risk.

To keep the analysis focused and interpretable, we examine five
representative numerical features that capture key behavioral and
financial dimensions of customer retention:

\begin{enumerate}
\def\labelenumi{\arabic{enumi}.}
\tightlist
\item
  \texttt{contacts.count.12}: number of customer service contacts in the
  past 12 months, reflecting engagement and potential dissatisfaction;\\
\item
  \texttt{transaction.amount.12}: total amount spent during the past 12
  months, representing spending activity;\\
\item
  \texttt{credit.limit}: total credit line assigned to each customer,
  indicating financial capacity;\\
\item
  \texttt{months.on.book}: number of months the customer has held the
  account, representing relationship duration;\\
\item
  \texttt{ratio.amount.Q4.Q1}: ratio of total transaction amount in the
  fourth quarter to that in the first quarter, capturing changes in
  spending behavior over time.
\end{enumerate}

Together, these variables provide a comprehensive view of customer
interaction, engagement, credit strength, tenure, and behavioral change.
After examining these features individually, the subsection
\emph{Assessing Correlation and Redundancy} explores how numerical
variables relate to one another and identifies potential overlaps. This
step ensures that later models are based on distinct, non-redundant
predictors that contribute meaningfully to churn analysis.

\subsection*{Customer Contacts and
Churn}\label{customer-contacts-and-churn}
\addcontentsline{toc}{subsection}{Customer Contacts and Churn}

The number of customer service contacts within the past year
(\texttt{contacts.count.12}) provides insight into customer engagement
and potential dissatisfaction. This variable is discrete, taking small
integer values that represent the number of times a customer contacted
support. For such count-based variables, bar plots are preferred to
boxplots or density plots, as they better convey frequency distributions
and group proportions.

\begin{Shaded}
\begin{Highlighting}[]
\FunctionTok{ggplot}\NormalTok{(}\AttributeTok{data =}\NormalTok{ churnCredit) }\SpecialCharTok{+}
  \FunctionTok{geom\_bar}\NormalTok{(}\FunctionTok{aes}\NormalTok{(}\AttributeTok{x =} \FunctionTok{factor}\NormalTok{(contacts.count}\FloatTok{.12}\NormalTok{), }\AttributeTok{fill =}\NormalTok{ churn)) }\SpecialCharTok{+}
  \FunctionTok{labs}\NormalTok{(}\AttributeTok{x =} \StringTok{"Number of Contacts in Past 12 Months"}\NormalTok{, }\AttributeTok{y =} \StringTok{"Count"}\NormalTok{)}

\FunctionTok{ggplot}\NormalTok{(}\AttributeTok{data =}\NormalTok{ churnCredit) }\SpecialCharTok{+}
  \FunctionTok{geom\_bar}\NormalTok{(}\FunctionTok{aes}\NormalTok{(}\AttributeTok{x =} \FunctionTok{factor}\NormalTok{(contacts.count}\FloatTok{.12}\NormalTok{), }\AttributeTok{fill =}\NormalTok{ churn), }\AttributeTok{position =} \StringTok{"fill"}\NormalTok{) }\SpecialCharTok{+}
  \FunctionTok{labs}\NormalTok{(}\AttributeTok{x =} \StringTok{"Number of Contacts in Past 12 Months"}\NormalTok{, }\AttributeTok{y =} \StringTok{"Proportion"}\NormalTok{)}
\end{Highlighting}
\end{Shaded}

\begin{figure}

\begin{minipage}{0.50\linewidth}
\begin{center}
\pandocbounded{\includegraphics[keepaspectratio]{4-Exploratory-data-analysis_files/figure-pdf/unnamed-chunk-19-1.pdf}}
\end{center}
\end{minipage}%
%
\begin{minipage}{0.50\linewidth}
\begin{center}
\pandocbounded{\includegraphics[keepaspectratio]{4-Exploratory-data-analysis_files/figure-pdf/unnamed-chunk-19-2.pdf}}
\end{center}
\end{minipage}%

\end{figure}%

The bar plots show that churn rates increase among customers who
contacted customer service more frequently, particularly those with four
or more interactions within the year. This pattern suggests that
repeated service contacts may reflect dissatisfaction or unresolved
issues.

From an analytical perspective, this feature offers an interpretable
behavioral signal. Customers who contact support frequently are more
likely to close their accounts, making this variable a useful early
indicator of churn risk.

\subsection*{Transaction Amount and
Churn}\label{transaction-amount-and-churn}
\addcontentsline{toc}{subsection}{Transaction Amount and Churn}

The total transaction amount in the past twelve months
(\texttt{transaction.amount.12}) reflects the level of customer activity
and engagement with the credit card. Higher transaction volumes
generally indicate active usage and satisfaction, while lower spending
may signal disengagement or a shift toward other financial products.

Unlike count-based variables such as \texttt{contacts.count.12},
\texttt{transaction.amount.12} is continuous, making boxplots and
density plots more suitable visualization tools. These plots summarize
the central tendency, spread, and distributional differences between
churned and active customers.

\begin{Shaded}
\begin{Highlighting}[]
\FunctionTok{ggplot}\NormalTok{(}\AttributeTok{data =}\NormalTok{ churnCredit) }\SpecialCharTok{+}
  \FunctionTok{geom\_boxplot}\NormalTok{(}\FunctionTok{aes}\NormalTok{(}\AttributeTok{x =}\NormalTok{ churn, }\AttributeTok{y =}\NormalTok{ transaction.amount}\FloatTok{.12}\NormalTok{), }
               \AttributeTok{fill =} \FunctionTok{c}\NormalTok{(}\StringTok{"\#F4A582"}\NormalTok{, }\StringTok{"\#A8D5BA"}\NormalTok{)) }\SpecialCharTok{+}
  \FunctionTok{labs}\NormalTok{(}\AttributeTok{x =} \StringTok{"Churn"}\NormalTok{, }\AttributeTok{y =} \StringTok{"Total Transaction Amount (12 months)"}\NormalTok{)}

\FunctionTok{ggplot}\NormalTok{(}\AttributeTok{data =}\NormalTok{ churnCredit) }\SpecialCharTok{+}
  \FunctionTok{geom\_density}\NormalTok{(}\FunctionTok{aes}\NormalTok{(}\AttributeTok{x =}\NormalTok{ transaction.amount}\FloatTok{.12}\NormalTok{, }\AttributeTok{fill =}\NormalTok{ churn), }\AttributeTok{alpha =} \FloatTok{0.6}\NormalTok{) }\SpecialCharTok{+}
  \FunctionTok{labs}\NormalTok{(}\AttributeTok{x =} \StringTok{"Total Transaction Amount (12 months)"}\NormalTok{, }\AttributeTok{y =} \StringTok{"Density"}\NormalTok{)}
\end{Highlighting}
\end{Shaded}

\begin{figure}

\begin{minipage}{0.50\linewidth}
\begin{center}
\pandocbounded{\includegraphics[keepaspectratio]{4-Exploratory-data-analysis_files/figure-pdf/unnamed-chunk-20-1.pdf}}
\end{center}
\end{minipage}%
%
\begin{minipage}{0.50\linewidth}
\begin{center}
\pandocbounded{\includegraphics[keepaspectratio]{4-Exploratory-data-analysis_files/figure-pdf/unnamed-chunk-20-2.pdf}}
\end{center}
\end{minipage}%

\end{figure}%

Both plots reveal a clear distinction between churners and non-churners.
Customers who churn tend to have significantly lower total transaction
amounts, indicating reduced engagement with the credit card over the
preceding year. In contrast, customers who remain active generally show
higher and more varied transaction volumes.

From a business standpoint, declining transaction activity can serve as
an early warning signal of disengagement. Monitoring spending behavior
and offering personalized incentives---such as cashback or loyalty
rewards---may help encourage continued use and improve retention among
low-activity customers.

\subsection*{Credit Limit and Churn}\label{credit-limit-and-churn}
\addcontentsline{toc}{subsection}{Credit Limit and Churn}

The total credit line assigned to each customer (\texttt{credit.limit})
reflects both financial capacity and the bank's confidence in the
customer's creditworthiness. Customers with higher credit limits
generally represent lower-risk or more established clients, who may also
be less likely to close their accounts.

Because \texttt{credit.limit} is a continuous variable that can vary
widely across customers, violin and histogram plots are effective tools
for visualizing its distribution by churn status. These plots reveal
differences in both median values and distribution shapes between the
two groups.

\begin{Shaded}
\begin{Highlighting}[]
\FunctionTok{ggplot}\NormalTok{(}\AttributeTok{data =}\NormalTok{ churnCredit, }\FunctionTok{aes}\NormalTok{(}\AttributeTok{x =}\NormalTok{ churn, }\AttributeTok{y =}\NormalTok{ credit.limit, }\AttributeTok{fill =}\NormalTok{ churn)) }\SpecialCharTok{+}
  \FunctionTok{geom\_violin}\NormalTok{(}\AttributeTok{trim =} \ConstantTok{FALSE}\NormalTok{) }\SpecialCharTok{+}
  \FunctionTok{labs}\NormalTok{(}\AttributeTok{x =} \StringTok{"Churn"}\NormalTok{, }\AttributeTok{y =} \StringTok{"Credit Limit"}\NormalTok{)}

\FunctionTok{ggplot}\NormalTok{(}\AttributeTok{data =}\NormalTok{ churnCredit) }\SpecialCharTok{+}
  \FunctionTok{geom\_histogram}\NormalTok{(}\FunctionTok{aes}\NormalTok{(}\AttributeTok{x =}\NormalTok{ credit.limit, }\AttributeTok{fill =}\NormalTok{ churn)) }\SpecialCharTok{+}
  \FunctionTok{labs}\NormalTok{(}\AttributeTok{x =} \StringTok{"Credit Limit"}\NormalTok{, }\AttributeTok{y =} \StringTok{"Count"}\NormalTok{)}
\end{Highlighting}
\end{Shaded}

\begin{figure}

\begin{minipage}{0.50\linewidth}
\begin{center}
\pandocbounded{\includegraphics[keepaspectratio]{4-Exploratory-data-analysis_files/figure-pdf/unnamed-chunk-21-1.pdf}}
\end{center}
\end{minipage}%
%
\begin{minipage}{0.50\linewidth}
\begin{center}
\pandocbounded{\includegraphics[keepaspectratio]{4-Exploratory-data-analysis_files/figure-pdf/unnamed-chunk-21-2.pdf}}
\end{center}
\end{minipage}%

\end{figure}%

Both plots show that customers who churn tend to have lower credit
limits compared to those who remain active. The distribution for
non-churners is more dispersed and shifted toward higher values,
suggesting that customers with greater financial capacity are less
inclined to close their accounts.

From a business perspective, this pattern implies that customers with
smaller credit limits might perceive limited benefits from maintaining
their credit card accounts. Offering credit line increases to eligible
customers, or designing products tailored to their spending capacity,
could help improve retention in this segment.

\subsection*{Months on Book and Churn}\label{months-on-book-and-churn}
\addcontentsline{toc}{subsection}{Months on Book and Churn}

The variable \texttt{months.on.book} measures the length of time a
customer has maintained their credit card account with the bank. Tenure
often reflects relationship stability and long-term engagement.
Customers with longer histories tend to be more loyal and may have
developed stronger trust or accumulated benefits, while newer customers
might be more likely to close their accounts if their expectations are
not met.

\begin{Shaded}
\begin{Highlighting}[]
\FunctionTok{ggplot}\NormalTok{(}\AttributeTok{data =}\NormalTok{ churnCredit, }\FunctionTok{aes}\NormalTok{(}\AttributeTok{x =}\NormalTok{ churn, }\AttributeTok{y =}\NormalTok{ months.on.book, }\AttributeTok{fill =}\NormalTok{ churn)) }\SpecialCharTok{+}
  \FunctionTok{geom\_violin}\NormalTok{(}\AttributeTok{alpha =} \FloatTok{0.5}\NormalTok{, }\AttributeTok{trim =} \ConstantTok{TRUE}\NormalTok{) }\SpecialCharTok{+}
  \FunctionTok{geom\_boxplot}\NormalTok{(}\AttributeTok{width =} \FloatTok{0.15}\NormalTok{, }\AttributeTok{fill =} \StringTok{"white"}\NormalTok{, }\AttributeTok{outlier.shape =} \ConstantTok{NA}\NormalTok{) }\SpecialCharTok{+}
  \FunctionTok{labs}\NormalTok{(}\AttributeTok{x =} \StringTok{"Churn"}\NormalTok{, }\AttributeTok{y =} \StringTok{"Months on Book"}\NormalTok{) }\SpecialCharTok{+}
  \FunctionTok{theme}\NormalTok{(}\AttributeTok{legend.position =} \StringTok{"none"}\NormalTok{)}

\FunctionTok{ggplot}\NormalTok{(}\AttributeTok{data =}\NormalTok{ churnCredit) }\SpecialCharTok{+}
  \FunctionTok{geom\_histogram}\NormalTok{(}\FunctionTok{aes}\NormalTok{(}\AttributeTok{x =}\NormalTok{ months.on.book, }\AttributeTok{fill =}\NormalTok{ churn), }\AttributeTok{bins =} \DecValTok{20}\NormalTok{) }\SpecialCharTok{+}
  \FunctionTok{labs}\NormalTok{(}\AttributeTok{x =} \StringTok{"Months on Book"}\NormalTok{, }\AttributeTok{y =} \StringTok{"Density"}\NormalTok{)}
\end{Highlighting}
\end{Shaded}

\begin{figure}

\begin{minipage}{0.50\linewidth}
\begin{center}
\pandocbounded{\includegraphics[keepaspectratio]{4-Exploratory-data-analysis_files/figure-pdf/unnamed-chunk-22-1.pdf}}
\end{center}
\end{minipage}%
%
\begin{minipage}{0.50\linewidth}
\begin{center}
\pandocbounded{\includegraphics[keepaspectratio]{4-Exploratory-data-analysis_files/figure-pdf/unnamed-chunk-22-2.pdf}}
\end{center}
\end{minipage}%

\end{figure}%

Both plots suggest that customers who churn tend to have slightly
shorter tenures than those who remain active. Although the difference is
not large, it is consistent: the median tenure of churners is a few
months lower. The noticeable peak around 36 months reflects a cohort
effect, likely linked to a major customer acquisition campaign that
occurred three years before the observation period.

From a business perspective, such findings can inform retention programs
targeting newer customers. Strengthening onboarding processes, enhancing
early engagement, or providing tailored incentives during the first year
could help increase customer lifetime value and reduce attrition.

\subsection*{Ratio of Transaction Amount (Q4/Q1) and
Churn}\label{ratio-of-transaction-amount-q4q1-and-churn}
\addcontentsline{toc}{subsection}{Ratio of Transaction Amount (Q4/Q1)
and Churn}

The variable \texttt{ratio.amount.Q4.Q1} measures the ratio between the
total transaction amount in the fourth quarter and that in the first
quarter. It captures how customer spending changes over time, offering a
behavioral perspective on financial activity. A declining ratio may
suggest reduced engagement or satisfaction, while an increasing ratio
could indicate growing usage and confidence in the service.

\begin{Shaded}
\begin{Highlighting}[]
\FunctionTok{ggplot}\NormalTok{(}\AttributeTok{data =}\NormalTok{ churnCredit) }\SpecialCharTok{+}
  \FunctionTok{geom\_boxplot}\NormalTok{(}\FunctionTok{aes}\NormalTok{(}\AttributeTok{x =}\NormalTok{ churn, }\AttributeTok{y =}\NormalTok{ ratio.amount.Q4.Q1), }
               \AttributeTok{fill =} \FunctionTok{c}\NormalTok{(}\StringTok{"\#F4A582"}\NormalTok{, }\StringTok{"\#A8D5BA"}\NormalTok{)) }\SpecialCharTok{+}
  \FunctionTok{labs}\NormalTok{(}\AttributeTok{x =} \StringTok{"Churn"}\NormalTok{, }\AttributeTok{y =} \StringTok{"Transaction Amount Ratio (Q4/Q1)"}\NormalTok{)}

\FunctionTok{ggplot}\NormalTok{(}\AttributeTok{data =}\NormalTok{ churnCredit) }\SpecialCharTok{+}
  \FunctionTok{geom\_density}\NormalTok{(}\FunctionTok{aes}\NormalTok{(}\AttributeTok{x =}\NormalTok{ ratio.amount.Q4.Q1, }\AttributeTok{fill =}\NormalTok{ churn), }\AttributeTok{alpha =} \FloatTok{0.6}\NormalTok{) }\SpecialCharTok{+}
  \FunctionTok{labs}\NormalTok{(}\AttributeTok{x =} \StringTok{"Transaction Amount Ratio (Q4/Q1)"}\NormalTok{, }\AttributeTok{y =} \StringTok{"Density"}\NormalTok{)}
\end{Highlighting}
\end{Shaded}

\begin{figure}

\begin{minipage}{0.50\linewidth}
\begin{center}
\pandocbounded{\includegraphics[keepaspectratio]{4-Exploratory-data-analysis_files/figure-pdf/unnamed-chunk-23-1.pdf}}
\end{center}
\end{minipage}%
%
\begin{minipage}{0.50\linewidth}
\begin{center}
\pandocbounded{\includegraphics[keepaspectratio]{4-Exploratory-data-analysis_files/figure-pdf/unnamed-chunk-23-2.pdf}}
\end{center}
\end{minipage}%

\end{figure}%

The plots reveal that customers who churn tend to have lower Q4-to-Q1
ratios, indicating a reduction in spending toward the end of the year.
In contrast, customers who remain active typically maintain or slightly
increase their transaction volumes between the two periods.

This decline in spending may serve as an early signal of disengagement:
customers who gradually reduce their credit card usage are more likely
to close their accounts later.

From a business perspective, monitoring quarterly spending patterns can
help identify at-risk customers. Proactive measures, such as targeted
rewards or seasonal incentives for reduced spenders, could mitigate
churn risk and sustain long-term engagement.

\begin{quote}
\emph{Try it yourself:} Repeat the same visual analysis using other
numerical features such as \texttt{age} and \texttt{months.inactive}. Do
you observe similar or different patterns of behavior between churners
and non-churners? Reflect on how these variables might contribute to
predicting customer retention.
\end{quote}

\subsection{Assessing Correlation and
Redundancy}\label{sec-ch4-EDA-correlation}

Before analyzing more complex interactions among variables, it is
helpful to assess how numerical features relate to one another.
Correlation analysis helps identify variables that may carry overlapping
information or exhibit redundancy. Recognizing such relationships early
simplifies the modeling process and reduces the risk of
multicollinearity.

Correlation quantifies how two variables move together. A positive
correlation means that as one variable increases, the other tends to
increase as well; a negative correlation suggests that one decreases as
the other increases. The Pearson correlation coefficient, denoted by
\(r\), summarizes this relationship on a scale from \(-1\) to \(1\). A
value of \(r = 1\) indicates a perfect positive relationship, \(r = -1\)
a perfect negative relationship, and \(r = 0\) no linear association.

\begin{figure}[H]

\centering{

\includegraphics[width=1\linewidth,height=\textheight,keepaspectratio]{images/ch4_correlation.png}

}

\caption{\label{fig-correlation}Example scatterplots showing different
correlation coefficients.}

\end{figure}%

\begin{quote}
\emph{Note:} Correlation does not imply causation. For example, a strong
positive correlation between customer contacts and churn does not mean
that contacting customer service causes customers to leave. Both
behaviors may stem from an underlying factor, such as dissatisfaction
with service.
\end{quote}

To illustrate this point, Figure \ref{fig-correlation-chocolate} shows a
well-known example from Messerli (2012), depicting a strong correlation
between per-capita chocolate consumption and Nobel Prize wins across
countries. Although amusing, it underscores the importance of caution:
correlations may arise by coincidence or through the influence of
unobserved factors. For readers interested in causality, \emph{The Book
of Why} by Judea Pearl and Dana Mackenzie (\textbf{pearl2018book?})
offers an accessible introduction to this topic.

\begin{figure}[H]

\centering{

\includegraphics[width=0.6\linewidth,height=\textheight,keepaspectratio]{images/ch4_correlation_chocolate.png}

}

\caption{\label{fig-correlation-chocolate}Scatterplot illustrating the
correlation between Nobel Prize wins and chocolate consumption (per 10
million population) across countries. Adapted from Messerli (2012).}

\end{figure}%

Returning to the \emph{churnCredit} dataset, we compute and visualize
the correlation matrix for all numerical variables using a heatmap. This
visualization helps detect redundant or related features before
modeling.

\begin{Shaded}
\begin{Highlighting}[]
\FunctionTok{library}\NormalTok{(ggcorrplot)}

\NormalTok{numeric\_features }\OtherTok{=} \FunctionTok{c}\NormalTok{(}\StringTok{"age"}\NormalTok{, }\StringTok{"dependent.count"}\NormalTok{, }\StringTok{"months.on.book"}\NormalTok{, }
             \StringTok{"relationship.count"}\NormalTok{, }\StringTok{"months.inactive"}\NormalTok{, }\StringTok{"contacts.count.12"}\NormalTok{, }
             \StringTok{"credit.limit"}\NormalTok{, }\StringTok{"revolving.balance"}\NormalTok{, }\StringTok{"available.credit"}\NormalTok{, }
             \StringTok{"transaction.amount.12"}\NormalTok{, }\StringTok{"transaction.count.12"}\NormalTok{, }
             \StringTok{"ratio.amount.Q4.Q1"}\NormalTok{, }\StringTok{"ratio.count.Q4.Q1"}\NormalTok{, }\StringTok{"utilization.ratio"}\NormalTok{)}

\NormalTok{cor\_matrix }\OtherTok{=} \FunctionTok{cor}\NormalTok{(churnCredit[, numeric\_features])}

\FunctionTok{ggcorrplot}\NormalTok{(cor\_matrix, }\AttributeTok{type =} \StringTok{"lower"}\NormalTok{, }\AttributeTok{lab =} \ConstantTok{TRUE}\NormalTok{, }\AttributeTok{lab\_size =} \DecValTok{2}\NormalTok{, }\AttributeTok{tl.cex =} \DecValTok{6}\NormalTok{, }
           \AttributeTok{colors =} \FunctionTok{c}\NormalTok{(}\StringTok{"\#699fb3"}\NormalTok{, }\StringTok{"white"}\NormalTok{, }\StringTok{"\#b3697a"}\NormalTok{),}
           \AttributeTok{title =} \StringTok{"Visualization of the Correlation Matrix"}\NormalTok{)}
\end{Highlighting}
\end{Shaded}

\begin{center}
\includegraphics[width=1\linewidth,height=\textheight,keepaspectratio]{4-Exploratory-data-analysis_files/figure-pdf/unnamed-chunk-24-1.pdf}
\end{center}

The heatmap shows that most numerical variables in the
\emph{churnCredit} dataset are only moderately or weakly correlated,
suggesting that they capture distinct behavioral dimensions. The
exception is the perfect correlation (\(r = 1\)) between
\texttt{credit.limit} and \texttt{available.credit}, meaning one is
mathematically derived from the other. Including both in a model would
therefore add redundancy without additional information. To illustrate
this visually:

\begin{Shaded}
\begin{Highlighting}[]
\FunctionTok{ggplot}\NormalTok{(}\AttributeTok{data =}\NormalTok{ churnCredit) }\SpecialCharTok{+}
    \FunctionTok{geom\_point}\NormalTok{(}\FunctionTok{aes}\NormalTok{(}\AttributeTok{x =}\NormalTok{ credit.limit, }\AttributeTok{y =}\NormalTok{ available.credit), }\AttributeTok{size =} \FloatTok{0.1}\NormalTok{) }\SpecialCharTok{+}
    \FunctionTok{labs}\NormalTok{(}\AttributeTok{x =} \StringTok{"Credit Limit"}\NormalTok{, }\AttributeTok{y =} \StringTok{"Available Credit"}\NormalTok{)}

\FunctionTok{ggplot}\NormalTok{(}\AttributeTok{data =}\NormalTok{ churnCredit) }\SpecialCharTok{+}
    \FunctionTok{geom\_point}\NormalTok{(}\FunctionTok{aes}\NormalTok{(}\AttributeTok{x =}\NormalTok{ credit.limit }\SpecialCharTok{{-}}\NormalTok{ revolving.balance, }
                   \AttributeTok{y =}\NormalTok{ available.credit), }\AttributeTok{size =} \FloatTok{0.1}\NormalTok{) }\SpecialCharTok{+}
    \FunctionTok{labs}\NormalTok{(}\AttributeTok{x =} \StringTok{"Credit Limit {-} Revolving Balance"}\NormalTok{, }\AttributeTok{y =} \StringTok{"Available Credit"}\NormalTok{)}
\end{Highlighting}
\end{Shaded}

\begin{figure}

\begin{minipage}{0.50\linewidth}
\begin{center}
\pandocbounded{\includegraphics[keepaspectratio]{4-Exploratory-data-analysis_files/figure-pdf/unnamed-chunk-25-1.pdf}}
\end{center}
\end{minipage}%
%
\begin{minipage}{0.50\linewidth}
\begin{center}
\pandocbounded{\includegraphics[keepaspectratio]{4-Exploratory-data-analysis_files/figure-pdf/unnamed-chunk-25-2.pdf}}
\end{center}
\end{minipage}%

\end{figure}%

The first plot shows a perfect linear relationship between
\texttt{credit.limit} and \texttt{available.credit}. The second confirms
that \texttt{available.credit} is essentially equal to
\texttt{credit.limit\ -\ revolving.balance}, validating the redundancy
observed in the correlation matrix.

A similar relationship is observed between \texttt{utilization.ratio},
\texttt{revolving.balance}, and \texttt{credit.limit}. The utilization
ratio is mathematically defined as
\texttt{revolving.balance\ /\ credit.limit}, meaning it does not
introduce new information but provides a normalized view of credit
usage. Depending on the modeling goal, it may be preferable to retain
either the ratio for interpretability or its components for more
detailed financial analysis. To illustrate this visually:

\begin{Shaded}
\begin{Highlighting}[]
\FunctionTok{ggplot}\NormalTok{(}\AttributeTok{data =}\NormalTok{ churnCredit) }\SpecialCharTok{+}
    \FunctionTok{geom\_point}\NormalTok{(}\FunctionTok{aes}\NormalTok{(}\AttributeTok{x =}\NormalTok{ credit.limit, }\AttributeTok{y =}\NormalTok{ utilization.ratio), }\AttributeTok{size =} \FloatTok{0.1}\NormalTok{) }\SpecialCharTok{+}
    \FunctionTok{labs}\NormalTok{(}\AttributeTok{x =} \StringTok{"Credit Limit"}\NormalTok{, }\AttributeTok{y =} \StringTok{"Utilization Ratio"}\NormalTok{)}

\FunctionTok{ggplot}\NormalTok{(}\AttributeTok{data =}\NormalTok{ churnCredit) }\SpecialCharTok{+}
    \FunctionTok{geom\_point}\NormalTok{(}\FunctionTok{aes}\NormalTok{(}\AttributeTok{x =}\NormalTok{ revolving.balance}\SpecialCharTok{/}\NormalTok{credit.limit, }
                   \AttributeTok{y =}\NormalTok{ utilization.ratio), }\AttributeTok{size =} \FloatTok{0.1}\NormalTok{) }\SpecialCharTok{+}
    \FunctionTok{labs}\NormalTok{(}\AttributeTok{x =} \StringTok{"Revolving Balance / Credit Limit"}\NormalTok{, }\AttributeTok{y =} \StringTok{"Utilization Ratio"}\NormalTok{)}
\end{Highlighting}
\end{Shaded}

\begin{figure}

\begin{minipage}{0.50\linewidth}
\begin{center}
\pandocbounded{\includegraphics[keepaspectratio]{4-Exploratory-data-analysis_files/figure-pdf/unnamed-chunk-26-1.pdf}}
\end{center}
\end{minipage}%
%
\begin{minipage}{0.50\linewidth}
\begin{center}
\pandocbounded{\includegraphics[keepaspectratio]{4-Exploratory-data-analysis_files/figure-pdf/unnamed-chunk-26-2.pdf}}
\end{center}
\end{minipage}%

\end{figure}%

Removing redundant variables (such as \texttt{utilization.ratio} or its
derived counterparts) simplifies the dataset and reduces potential
multicollinearity. The remaining variables exhibit mostly low or
moderate correlations, indicating that they provide complementary
information about customer behavior.

\subsection*{Summary of Numerical
Findings}\label{summary-of-numerical-findings}
\addcontentsline{toc}{subsection}{Summary of Numerical Findings}

The exploratory analysis of numerical features in the \emph{churnCredit}
dataset reveals several key behavioral and financial patterns:

\begin{itemize}
\item
  Customers with fewer customer service contacts tend to remain active,
  whereas those with multiple interactions are more likely to churn,
  suggesting a link between frequent service issues and dissatisfaction.
\item
  Spending activity, measured by \texttt{transaction.amount.12}, shows a
  strong association with retention. Customers who spend more are
  generally more engaged and less likely to close their accounts.
\item
  Credit-related variables such as \texttt{credit.limit} and
  \texttt{available.credit} are perfectly correlated, capturing the same
  financial capacity information. Similarly, the
  \texttt{utilization.ratio} is mathematically derived from
  \texttt{revolving.balance} and \texttt{credit.limit}. Retaining only
  one of these measures avoids redundancy and improves model clarity.
\item
  The variable \texttt{months.on.book} indicates that customers with
  longer relationships are slightly less likely to churn, though the
  difference is modest. Early-stage customers may require targeted
  onboarding and retention programs.
\item
  The \texttt{ratio.amount.Q4.Q1} feature suggests that a decline in
  spending between the first and fourth quarters often precedes account
  closure, highlighting potential seasonal disengagement patterns.
\item
  Correlation analysis confirmed that, apart from these derived or
  perfectly correlated variables, most numerical features are only
  moderately or weakly correlated, meaning they provide distinct
  insights into customer behavior.
\end{itemize}

Overall, the numerical analysis highlights that customer churn is more
strongly influenced by behavioral and financial indicators---such as
spending activity, utilization, and engagement---than by demographic
characteristics. These findings provide a quantitative foundation for
feature selection and modeling in the following chapters.

\section{Exploring Multivariate
Relationships}\label{sec-EDA-sec-multivariate}

Univariate and pairwise analyses provide important context, but
real-world customer behaviour often emerges from the interaction of
multiple variables. This section examines how financial activity and
demographic factors combine to explain churn, helping identify customer
groups with distinct risk profiles.

By analysing combinations of variables, such as joint spending and
usage, or spending differences across card categories, we obtain a more
complete view of how customers engage with their accounts and how this
relates to account closure.

\subsection*{Joint Patterns in Transaction Amount and
Count}\label{joint-patterns-in-transaction-amount-and-count}
\addcontentsline{toc}{subsection}{Joint Patterns in Transaction Amount
and Count}

Transaction activity has two important dimensions: how much customers
spend and how often they use their card. The variables
\texttt{transaction.amount.12} and \texttt{transaction.count.12}
represent these behaviours over a twelve-month period. Examining them
together provides insight into usage patterns that are not visible in
univariate plots. A scatter plot, enhanced with marginal histograms, is
particularly useful here because it displays both the joint pattern and
the separate distributions of each variable.

The code below first constructs a base scatter plot using
\textbf{ggplot2} and then applies the function \texttt{ggMarginal()}
from the \textbf{ggExtra} package to add histograms along the horizontal
and vertical axes:

\begin{Shaded}
\begin{Highlighting}[]
\FunctionTok{library}\NormalTok{(ggExtra)}

\CommentTok{\# Base scatter plot}
\NormalTok{scatter\_plot }\OtherTok{\textless{}{-}} \FunctionTok{ggplot}\NormalTok{(}\AttributeTok{data =}\NormalTok{ churnCredit) }\SpecialCharTok{+}
  \FunctionTok{geom\_point}\NormalTok{(}\FunctionTok{aes}\NormalTok{(}\AttributeTok{x =}\NormalTok{ transaction.amount}\FloatTok{.12}\NormalTok{, }\AttributeTok{y =}\NormalTok{ transaction.count}\FloatTok{.12}\NormalTok{, }
                 \AttributeTok{color =}\NormalTok{ churn), }\AttributeTok{size =} \FloatTok{0.2}\NormalTok{, }\AttributeTok{alpha =} \FloatTok{0.7}\NormalTok{) }\SpecialCharTok{+}
  \FunctionTok{labs}\NormalTok{(}\AttributeTok{x =} \StringTok{"Transaction Amount"}\NormalTok{, }\AttributeTok{y =} \StringTok{"Total Transaction Count (12 months)"}\NormalTok{) }\SpecialCharTok{+}
  \FunctionTok{theme}\NormalTok{(}\AttributeTok{legend.position =} \StringTok{"bottom"}\NormalTok{)}

\CommentTok{\# Add marginal histograms}
\FunctionTok{ggMarginal}\NormalTok{(scatter\_plot, }\AttributeTok{type =} \StringTok{"histogram"}\NormalTok{, }\AttributeTok{groupColour =} \ConstantTok{TRUE}\NormalTok{, }
           \AttributeTok{groupFill =} \ConstantTok{TRUE}\NormalTok{, }\AttributeTok{alpha =} \FloatTok{0.6}\NormalTok{, }\AttributeTok{size =} \DecValTok{4}\NormalTok{)}
\end{Highlighting}
\end{Shaded}

\begin{center}
\includegraphics[width=0.9\linewidth,height=\textheight,keepaspectratio]{4-Exploratory-data-analysis_files/figure-pdf/unnamed-chunk-27-1.pdf}
\end{center}

The central scatter plot shows a clear positive association: customers
who spend more also tend to make more transactions. Most observations
fall within a broad diagonal region of moderate spending and transaction
counts, where churners and non-churners largely overlap. The marginal
histograms summarise the distributions of \texttt{transaction.amount.12}
and \texttt{transaction.count.12} separately, offering a quick
comparison of usage patterns between churned and active accounts.

\begin{quote}
\emph{Try it yourself:} Replace \texttt{type\ =\ "histogram"} with
\texttt{type\ =\ "density"} in the \texttt{ggMarginal()} function to add
marginal density curves instead of histograms. Then repeat the scatter
plot using \texttt{ratio.amount.Q4.Q1} on the horizontal axis instead of
\texttt{transaction.amount.12}. Compare the resulting plots: which
combination of visualisation type and variables makes differences
between churners and non-churners easier to see?
\end{quote}

\begin{quote}
\emph{Optional exploration:} You may also wish to explore a 3D
perspective on these variables. The \textbf{plotly} package allows
interactive rotation and zooming, which can reveal patterns that may be
harder to detect in two-dimensional graphics. The following code
produces a 3D scatter plot in HTML output or in your editor (such as
RStudio), but it will not render in the PDF version of this book:
\end{quote}

\begin{Shaded}
\begin{Highlighting}[]
\FunctionTok{library}\NormalTok{(plotly)}

\FunctionTok{plot\_ly}\NormalTok{(}
  \AttributeTok{data =}\NormalTok{ churnCredit,}
  \AttributeTok{x =} \SpecialCharTok{\textasciitilde{}}\NormalTok{transaction.amount}\FloatTok{.12}\NormalTok{,}
  \AttributeTok{y =} \SpecialCharTok{\textasciitilde{}}\NormalTok{transaction.count}\FloatTok{.12}\NormalTok{,}
  \AttributeTok{z =} \SpecialCharTok{\textasciitilde{}}\NormalTok{credit.limit,}
  \AttributeTok{color =} \SpecialCharTok{\textasciitilde{}}\NormalTok{churn,}
  \AttributeTok{type =} \StringTok{"scatter3d"}\NormalTok{,}
  \AttributeTok{mode =} \StringTok{"markers"}\NormalTok{,}
  \AttributeTok{marker =} \FunctionTok{list}\NormalTok{(}\AttributeTok{size =} \DecValTok{2}\NormalTok{)}
\NormalTok{)}
\end{Highlighting}
\end{Shaded}

To examine specific patterns more closely, we focus on two illustrative
regions: customers with very low spending and those with moderate
spending but relatively few transactions. These regions are extracted
using the \texttt{subset()} function as follows:

\begin{Shaded}
\begin{Highlighting}[]
\NormalTok{sub\_churnCredit }\OtherTok{=} \FunctionTok{subset}\NormalTok{(}
\NormalTok{  churnCredit,}
\NormalTok{  (transaction.amount}\FloatTok{.12} \SpecialCharTok{\textless{}} \DecValTok{1000}\NormalTok{) }\SpecialCharTok{|}
\NormalTok{    ((}\DecValTok{2000} \SpecialCharTok{\textless{}}\NormalTok{ transaction.amount}\FloatTok{.12}\NormalTok{) }\SpecialCharTok{\&} 
\NormalTok{     (transaction.amount}\FloatTok{.12} \SpecialCharTok{\textless{}} \DecValTok{3000}\NormalTok{) }\SpecialCharTok{\&} 
\NormalTok{     (transaction.count}\FloatTok{.12} \SpecialCharTok{\textless{}} \DecValTok{52}\NormalTok{))}
\NormalTok{)}

\FunctionTok{ggplot}\NormalTok{(}\AttributeTok{data =}\NormalTok{ sub\_churnCredit, }
       \FunctionTok{aes}\NormalTok{(}\AttributeTok{x =}\NormalTok{ churn, }
           \AttributeTok{label =}\NormalTok{ scales}\SpecialCharTok{::}\FunctionTok{percent}\NormalTok{(}\FunctionTok{prop.table}\NormalTok{(}\FunctionTok{after\_stat}\NormalTok{(count))))) }\SpecialCharTok{+}
  \FunctionTok{geom\_bar}\NormalTok{(}\AttributeTok{fill =} \FunctionTok{c}\NormalTok{(}\StringTok{"\#F4A582"}\NormalTok{, }\StringTok{"\#A8D5BA"}\NormalTok{)) }\SpecialCharTok{+} 
  \FunctionTok{geom\_text}\NormalTok{(}\AttributeTok{stat =} \StringTok{"count"}\NormalTok{, }\AttributeTok{vjust =} \FloatTok{0.4}\NormalTok{, }\AttributeTok{size =} \DecValTok{8}\NormalTok{) }
\end{Highlighting}
\end{Shaded}

\begin{center}
\includegraphics[width=0.45\linewidth,height=\textheight,keepaspectratio]{4-Exploratory-data-analysis_files/figure-pdf/unnamed-chunk-29-1.pdf}
\end{center}

Within this subset, the proportion of churners is noticeably higher than
in the full dataset. These patterns suggest that low or inconsistent
card usage, particularly when combined with infrequent transactions, is
associated with an increased likelihood of churn.

From a modelling perspective, this example illustrates how interactions
between predictors can be important. Neither transaction amount nor
transaction count alone identifies these segments, but their combination
highlights customers with elevated churn risk. From a business
perspective, customers with low activity represent an opportunity for
targeted re-engagement, for example through usage incentives or
personalised communication.

\subsection*{Card Category and Spending
Patterns}\label{card-category-and-spending-patterns}
\addcontentsline{toc}{subsection}{Card Category and Spending Patterns}

The variable \texttt{card.category} divides customers into four product
tiers (blue, silver, gold, and platinum). The variable
\texttt{transaction.amount.12} measures the total amount spent over the
past twelve months. Examining these variables together provides insight
into how card tier relates to spending behaviour.

Because \texttt{transaction.amount.12} is continuous and
\texttt{card.category} is categorical, density plots are a natural
choice. They highlight differences in the shape, centre, and spread of
spending distributions across card tiers.

\begin{Shaded}
\begin{Highlighting}[]
\FunctionTok{ggplot}\NormalTok{(}\AttributeTok{data =}\NormalTok{ churnCredit, }\FunctionTok{aes}\NormalTok{(}\AttributeTok{x =}\NormalTok{ transaction.amount}\FloatTok{.12}\NormalTok{, }\AttributeTok{fill =}\NormalTok{ card.category)) }\SpecialCharTok{+}
  \FunctionTok{geom\_density}\NormalTok{(}\AttributeTok{alpha =} \FloatTok{0.5}\NormalTok{) }\SpecialCharTok{+}
  \FunctionTok{labs}\NormalTok{(}\AttributeTok{x =} \StringTok{"Total Transaction Amount (12 months)"}\NormalTok{,}
       \AttributeTok{y =} \StringTok{"Density"}\NormalTok{,}
       \AttributeTok{fill =} \StringTok{"Card Category"}\NormalTok{) }\SpecialCharTok{+}
  \FunctionTok{scale\_fill\_manual}\NormalTok{(}\AttributeTok{values =} \FunctionTok{c}\NormalTok{(}\StringTok{"\#1E90FF"}\NormalTok{, }\StringTok{"\#C0C0C0"}\NormalTok{, }\StringTok{"\#FFD700"}\NormalTok{, }\StringTok{"\#E5E4E2"}\NormalTok{))}
\end{Highlighting}
\end{Shaded}

\begin{center}
\includegraphics[width=0.65\linewidth,height=\textheight,keepaspectratio]{4-Exploratory-data-analysis_files/figure-pdf/unnamed-chunk-30-1.pdf}
\end{center}

The density curves show that customers with higher-tier cards,
particularly gold and platinum, tend to have higher transaction amounts.
Their distributions are shifted to the right relative to those of blue
and silver cardholders. Blue card customers are more concentrated in the
lower and middle spending ranges.

It is helpful to note that blue cards represent more than 90 percent of
all accounts, so the density curves reflect both within-group variation
and substantial differences in group size. Despite this imbalance, the
overall pattern is consistent: premium cards are associated with higher
spending levels.

From a business perspective, this relationship suggests that card tier
is closely linked to customer engagement and perceived value. Premium
cardholders tend to be high-value, active customers, while blue
cardholders form a more diverse group that includes many low-activity
accounts. These observations can inform differentiated retention and
marketing strategies, such as targeted upgrades or benefits aimed at
increasing engagement among standard-tier customers.

\subsection*{Transaction Analysis by
Age}\label{transaction-analysis-by-age}
\addcontentsline{toc}{subsection}{Transaction Analysis by Age}

Age is an important demographic factor that often shapes financial
behaviour, spending patterns, and loyalty to service providers. In the
\emph{churnCredit} dataset, understanding how transaction activity
varies across age groups can shed light on whether younger and older
customers differ in their likelihood of closing their accounts. Because
individual observations form a dense cloud, we use smoothed trend lines
to highlight overall patterns in spending and transaction frequency.

\begin{Shaded}
\begin{Highlighting}[]
\CommentTok{\# Total Transaction Amount by Age}
\FunctionTok{ggplot}\NormalTok{(}\AttributeTok{data =}\NormalTok{ churnCredit, }
       \FunctionTok{aes}\NormalTok{(}\AttributeTok{x =}\NormalTok{ age, }\AttributeTok{y =}\NormalTok{ transaction.amount}\FloatTok{.12}\NormalTok{, }\AttributeTok{color =}\NormalTok{ churn)) }\SpecialCharTok{+}
  \FunctionTok{geom\_smooth}\NormalTok{(}\AttributeTok{se =} \ConstantTok{FALSE}\NormalTok{, }\AttributeTok{linewidth =} \FloatTok{1.1}\NormalTok{, }\AttributeTok{alpha =} \FloatTok{0.9}\NormalTok{) }\SpecialCharTok{+}
  \FunctionTok{labs}\NormalTok{(}\AttributeTok{x =} \StringTok{"Customer Age"}\NormalTok{, }\AttributeTok{y =} \StringTok{"Total Transaction Amount (12 months)"}\NormalTok{) }

\CommentTok{\# Total Transaction Count by Age}
\FunctionTok{ggplot}\NormalTok{(}\AttributeTok{data =}\NormalTok{ churnCredit, }
       \FunctionTok{aes}\NormalTok{(}\AttributeTok{x =}\NormalTok{ age, }\AttributeTok{y =}\NormalTok{ transaction.count}\FloatTok{.12}\NormalTok{, }\AttributeTok{color =}\NormalTok{ churn)) }\SpecialCharTok{+}
  \FunctionTok{geom\_smooth}\NormalTok{(}\AttributeTok{se =} \ConstantTok{FALSE}\NormalTok{, }\AttributeTok{linewidth =} \FloatTok{1.1}\NormalTok{, }\AttributeTok{alpha =} \FloatTok{0.9}\NormalTok{) }\SpecialCharTok{+}
  \FunctionTok{labs}\NormalTok{(}\AttributeTok{x =} \StringTok{"Customer Age"}\NormalTok{, }\AttributeTok{y =} \StringTok{"Total Transaction Count (12 months)"}\NormalTok{) }
\end{Highlighting}
\end{Shaded}

\begin{figure}

\begin{minipage}{0.50\linewidth}
\begin{center}
\pandocbounded{\includegraphics[keepaspectratio]{4-Exploratory-data-analysis_files/figure-pdf/unnamed-chunk-31-1.pdf}}
\end{center}
\end{minipage}%
%
\begin{minipage}{0.50\linewidth}
\begin{center}
\pandocbounded{\includegraphics[keepaspectratio]{4-Exploratory-data-analysis_files/figure-pdf/unnamed-chunk-31-2.pdf}}
\end{center}
\end{minipage}%

\end{figure}%

The smooth curves indicate that transaction activity gradually declines
with age. Younger customers tend to make more transactions and spend
larger amounts, whereas older customers show lower and more stable
activity levels. There is also a small indication that churn is slightly
more common among younger, highly active customers, although this
difference is not large.

These patterns illustrate that activity level alone does not determine
churn. Instead, demographic factors such as age interact with
behavioural indicators to shape retention dynamics. Placing age
alongside spending and engagement measures gives a more comprehensive
view of customer behaviour than any single variable can provide.

\subsection*{Summary of Multivariate
Insights}\label{summary-of-multivariate-insights}
\addcontentsline{toc}{subsection}{Summary of Multivariate Insights}

The multivariate analysis of the \emph{churnCredit} dataset highlights
several interactions among behavioural, transactional, and demographic
features. These patterns reveal customer segments that do not emerge
through univariate exploration alone.

\begin{itemize}
\item
  Spending behaviour varies across card types. Higher-tier cardholders
  (gold and platinum) tend to spend more and exhibit lower churn rates,
  while blue card customers show wider variation in spending. This
  suggests that product tier is closely linked to engagement and
  perceived value.
\item
  The joint interpretation of transaction amount and transaction count
  provides insight into usage patterns that neither variable captures on
  its own. Customers with both low spending and infrequent transactions
  show noticeably higher churn rates, indicating potential disengagement
  or limited perceived benefit.
\item
  Younger customers generally make more transactions and spend larger
  amounts but exhibit slightly higher churn rates than older customers
  with similar activity levels. This is consistent with broader trends
  showing that younger customers may be more responsive to alternative
  financial products.
\item
  These interaction patterns show that churn risk cannot be inferred
  from a single behavioural indicator. Instead, combinations of
  financial activity, usage consistency, and demographic context provide
  a more complete understanding of retention dynamics.
\end{itemize}

Overall, the multivariate exploration indicates that customer churn
arises from the interplay of several behavioural and demographic
dimensions rather than isolated factors. These insights set the stage
for the modelling chapters that follow, where such interactions can be
quantified and used to improve predictive performance.

\section{Insights from EDA on Customer Churn}\label{sec-EDA-summary}

The exploratory analysis of the \emph{churnCredit} dataset has revealed
a multifaceted view of customer behavior and the factors associated with
churn. By examining both categorical and numerical variables, and how
they interact, it became clear that several patterns are relevant for
understanding and modelling customer attrition.

Churn behavior appears to be influenced by a combination of financial
engagement, credit usage, and service interaction rather than by
demographic attributes alone. Customers with lower credit limits, higher
utilization ratios, and larger revolving balances are more likely to
leave the service. This pattern suggests that financial pressure or
perceived credit constraints may contribute to dissatisfaction and
eventual account closure.

Behavioural engagement is another important aspect. Customers who
contacted customer service frequently, particularly those with four or
more interactions within a year, exhibit higher churn rates. This
tendency aligns with findings from other service sectors, where repeated
contact often indicates unresolved issues or growing dissatisfaction.
Similarly, customers who have remained with the bank for several years
but have recently become inactive may be at risk of leaving. These
individuals represent a valuable opportunity for reactivation.

Transaction activity provides additional insight into the link between
engagement and retention. Customers with higher transaction amounts and
more frequent transactions tend to remain active, highlighting the role
of product usage in sustaining customer relationships. When demographic
information is considered, a further pattern emerges: younger, highly
active customers churn slightly more often than older customers with
similar activity levels. This may reflect lower brand loyalty or a
greater openness to alternative financial products among younger
segments.

The analysis also highlights the influence of card type and customer
segmentation. Holders of higher-tier cards, such as gold or platinum,
generally show greater transaction activity and lower churn rates, which
suggests that additional benefits and perceived value support customer
loyalty. In contrast, customers with standard cards or smaller credit
limits may require targeted communication or loyalty initiatives to
strengthen their connection with the bank.

The exploratory work also revealed two instances of redundancy. The
variable \texttt{available.credit} is determined directly by subtracting
the revolving balance from the credit limit. Likewise, the utilization
ratio is calculated from the revolving balance and the credit limit.
Because these features do not introduce new information beyond their
component variables, they will be excluded when preparing the data for
modelling in Chapter \textbf{?@sec-ch7-classification-knn}. This
illustrates how EDA supports feature selection by identifying variables
that may complicate a model without improving predictive performance.

Overall, this exploratory analysis shows how systematic EDA can reveal
both structural characteristics of the data and meaningful behavioural
patterns. By integrating financial indicators, engagement measures, and
demographic profiles, the analysis provides a sound empirical basis for
the statistical inference and predictive modelling developed in the
following chapter. In Chapter \textbf{?@sec-ch5-statistics}, we will
revisit several of these patterns using formal statistical tests to
evaluate whether the differences observed in EDA reflect genuine
population-level effects.

\section{Chapter Summary and Takeaways}\label{sec-ch4-summary}

This chapter illustrated how exploratory data analysis supports the
transition from raw data to statistical modelling. Using the
\emph{churnCredit} dataset, we applied graphical and numerical
techniques to examine the structure of the data, identify potential
issues, and develop hypotheses about customer behaviour.

The analysis began with an overview of the dataset and an initial
preparation step, where missing values encoded as ``unknown'' were
identified and resolved. Ensuring that variables were clean and
correctly typed created a sound basis for exploration. The chapter then
moved through a systematic examination of categorical and numerical
features. Categorical variables such as gender, education, marital
status, income, and card type helped describe customer profiles, while
numerical variables such as credit limit, transaction activity, and
utilisation ratio highlighted financial and behavioural patterns.

The exploratory results revealed several consistent relationships.
Customers with smaller credit limits, higher utilisation ratios, or
frequent customer service interactions were more likely to churn. In
contrast, customers with higher transaction amounts and lower
utilisation levels tended to remain active. These observations
demonstrate how EDA can point toward potential explanatory variables
before any formal modelling is undertaken.

Multivariate exploration further showed how combinations of features
interact to shape churn behaviour. Joint patterns in transaction amount
and count, connections between card category and spending, and links
between age and financial activity illustrated that churn often arises
from a combination of behavioural, financial, and demographic factors
rather than isolated characteristics.

The chapter also highlighted the importance of identifying redundant
features. For instance, the variables available credit and utilisation
ratio were found to be deterministically related to other variables in
the dataset. Recognising such redundancy simplifies later modelling and
improves interpretability.

Taken together, the examples emphasise three guiding principles for
effective exploratory analysis. First, graphical and numerical summaries
should be used together to obtain complementary insights. Second,
careful attention to data quality, including missing values and
redundant variables, is essential for reliable analysis. Third, EDA is
not only descriptive: it provides direction for statistical inference
and predictive modelling by revealing patterns worth investigating
further.

The results of this chapter form the empirical foundation for the next
stage of the analysis. Chapter \textbf{?@sec-ch5-statistics} introduces
the tools of statistical inference, which allow us to formalise
uncertainty, quantify relationships, and test hypotheses suggested by
the exploratory findings.

\section{Exercises}\label{sec-ch4-exercises}

This section provides exercises designed to consolidate the concepts and
techniques introduced in this chapter. The questions cover conceptual
understanding, hands-on data exploration, and integrative challenges.
Exercises begin with short interpretive questions, followed by applied
analysis using the \emph{churn} and \emph{bank} datasets, and conclude
with advanced problems that encourage synthesis and critical reflection.

\subsubsection*{Conceptual Questions}\label{conceptual-questions}
\addcontentsline{toc}{subsubsection}{Conceptual Questions}

\begin{enumerate}
\def\labelenumi{\arabic{enumi}.}
\item
  Why is exploratory data analysis essential before building predictive
  models? What risks might arise if this step is skipped?
\item
  If a variable does not show a clear relationship with the target
  during EDA, should it be excluded from modeling? Consider potential
  interactions, hidden effects, and the role of feature selection.
\item
  What does it mean for two variables to be correlated? Explain the
  direction and strength of correlation, and contrast correlation with
  causation using an example.
\item
  How can correlated predictors be detected and addressed during EDA?
  Describe how this improves model performance and interpretability.
\item
  What are the potential consequences of including highly correlated
  variables in a predictive model? Discuss the effects on accuracy,
  interpretability, and model stability.
\item
  Is it always advisable to remove one of two correlated predictors?
  Under what circumstances might keeping both be justified?
\item
  For each of the following methods---histograms, box plots, density
  plots, scatter plots, summary statistics, correlation matrices,
  contingency tables, and bar plots---indicate whether it applies to
  categorical data, numerical data, or both. Briefly describe its role
  in EDA.
\item
  A bank observes that customers with high credit utilization and
  frequent customer service interactions are more likely to close their
  accounts. What actions could the bank take in response, and how might
  this guide retention strategy?
\item
  Suppose several pairs of variables in a dataset have high correlation
  (for example, \(r > 0.9\)). How would you handle this to ensure robust
  and interpretable modeling?
\item
  Why is it important to consider both statistical and practical
  relevance when evaluating correlations? Provide an example of a
  statistically strong but practically weak correlation.
\item
  Why is it important to investigate multivariate relationships in EDA?
  Describe a case where an interaction between two variables reveals a
  pattern that univariate analysis would miss.
\item
  How does data visualization support EDA? Provide two specific examples
  where visual tools reveal insights that summary statistics might
  obscure.
\item
  Suppose you discover that customers with both high credit utilization
  and frequent service calls are more likely to churn. What business
  strategies might be informed by this finding?
\item
  What are some common causes of outliers in data? How would you decide
  whether to retain, modify, or exclude an outlier?
\item
  Why is it important to address missing values during EDA? Discuss
  strategies for handling missing data and when each might be
  appropriate.
\end{enumerate}

\subsubsection*{\texorpdfstring{Hands-On Practice: Exploring the
\emph{churn}
Dataset}{Hands-On Practice: Exploring the churn Dataset}}\label{hands-on-practice-exploring-the-churn-dataset}
\addcontentsline{toc}{subsubsection}{Hands-On Practice: Exploring the
\emph{churn} Dataset}

The \emph{churn} dataset from the R package \textbf{liver} contains
information about customer behavior and service usage in a
telecommunications company. The objective is to identify patterns
associated with customer churn---whether a customer has left the
service. This dataset was also explored earlier in this chapter and will
be revisited in later chapters for classification modeling. This dataset
will be used for classification in the case study of Chapter
\textbf{?@sec-ch10-regression}. More details are available at
\url{https://rdrr.io/cran/liver/man/churn.html}.

To load and inspect the dataset:

\begin{Shaded}
\begin{Highlighting}[]
\FunctionTok{library}\NormalTok{(liver)}

\FunctionTok{data}\NormalTok{(churn)}
\FunctionTok{str}\NormalTok{(churn)}
\end{Highlighting}
\end{Shaded}

\begin{enumerate}
\def\labelenumi{\arabic{enumi}.}
\setcounter{enumi}{15}
\item
  Summarize the structure of the dataset and identify variable types.
  What information does this provide about the nature of the data?
\item
  Examine the target variable \texttt{churn}. What proportion of
  customers have left the service?
\item
  Explore the relationship between \texttt{intl.plan} and
  \texttt{churn}. Use bar plots and contingency tables to describe what
  you find.
\item
  Analyze the distribution of \texttt{customer.calls}. Which values
  occur most frequently? What might this indicate about customer
  engagement or dissatisfaction?
\item
  Investigate whether customers with higher \texttt{day.mins} are more
  likely to churn. Use box plots or density plots to support your
  reasoning.
\item
  Compute the correlation matrix for all numerical variables. Which
  features show strong relationships, and which appear independent?
\item
  Summarize your main EDA findings. What patterns emerge that could be
  relevant for predicting churn?
\item
  Reflect on business implications. Which customer behaviors appear most
  strongly associated with churn, and how could these insights inform a
  retention strategy?
\end{enumerate}

\subsubsection*{\texorpdfstring{Hands-On Practice: Exploring the
\emph{bank}
Dataset}{Hands-On Practice: Exploring the bank Dataset}}\label{hands-on-practice-exploring-the-bank-dataset}
\addcontentsline{toc}{subsubsection}{Hands-On Practice: Exploring the
\emph{bank} Dataset}

The \emph{bank} dataset from the R package \textbf{liver} contains data
on direct marketing campaigns of a Portuguese bank. The objective is to
predict whether a client subscribes to a term deposit. This dataset will
be used for classification in the case study of Chapter
\textbf{?@sec-ch12-neural-networks}. More details are available at
\url{https://rdrr.io/cran/liver/man/bank.html}.

To load and inspect the dataset:

\begin{Shaded}
\begin{Highlighting}[]
\FunctionTok{library}\NormalTok{(liver)}
\FunctionTok{data}\NormalTok{(bank)}
\FunctionTok{str}\NormalTok{(bank)}
\end{Highlighting}
\end{Shaded}

\begin{enumerate}
\def\labelenumi{\arabic{enumi}.}
\setcounter{enumi}{23}
\item
  Summarize the structure and variable types. What does this reveal
  about the dataset?
\item
  Plot the target variable \texttt{deposit}. What proportion of clients
  subscribed to a term deposit?
\item
  Explore the variables \texttt{default}, \texttt{housing}, and
  \texttt{loan} using bar plots and contingency tables. What patterns
  emerge?
\item
  Visualize the distributions of numerical features using histograms and
  box plots. Note any skewness or unusual observations.
\item
  Identify outliers among numerical variables. What strategies would you
  consider for handling them?
\item
  Compute and visualize correlations among numerical variables. Which
  features are highly correlated, and how might this influence modeling
  decisions?
\item
  Summarize your main EDA observations. How would you present these
  results in a report?
\item
  Interpret your findings in business terms. What actionable conclusions
  could the bank draw from these patterns?
\item
  Examine whether higher values of \texttt{campaign} (number of
  contacts) relate to greater subscription rates. Visualize and
  interpret.
\item
  Propose one new variable that could improve model performance based on
  your EDA findings.
\item
  Investigate subscription rates by \texttt{month}. Are some months more
  successful than others?
\item
  Explore how \texttt{job} relates to \texttt{deposit}. Which
  occupational groups have higher success rates?
\item
  Analyze the joint impact of \texttt{education} and \texttt{job} on
  subscription outcomes. What patterns do you observe?
\item
  Examine whether the \texttt{duration} of the last contact influences
  the likelihood of a positive outcome.
\item
  Compare success rates across campaigns. What strategies might these
  differences suggest?
\end{enumerate}

\subsubsection*{Challenge Problems}\label{challenge-problems}
\addcontentsline{toc}{subsubsection}{Challenge Problems}

\begin{enumerate}
\def\labelenumi{\arabic{enumi}.}
\setcounter{enumi}{38}
\item
  Create a concise one- or two-plot summary of an EDA finding from the
  \emph{bank} dataset. Focus on clarity and accessibility for a
  non-technical audience, using brief annotations to explain the
  insight.
\item
  Using the \emph{adult} dataset, identify a subgroup likely to earn
  over \$50K. Describe their characteristics and how you uncovered them
  through EDA.
\item
  A variable appears weakly related to the target in univariate plots.
  Under what conditions could it still improve model accuracy?
\item
  Examine whether the proportion of \texttt{deposit} outcomes differs by
  \texttt{marital} status or \texttt{job} category. What hypotheses
  could you draw from these differences?
\item
  Using the \emph{adult} dataset, identify predictors that may not
  contribute meaningfully to modeling. Justify your selections with
  evidence from EDA.
\end{enumerate}

\subsubsection*{Self-Reflection}\label{self-reflection}
\addcontentsline{toc}{subsubsection}{Self-Reflection}

Reflect on what you have learned in this chapter. Consider the following
questions as a guide:

\begin{itemize}
\tightlist
\item
  How has exploratory data analysis changed your understanding of the
  dataset before modeling?
\item
  Which visualizations or summary techniques did you find most effective
  for revealing structure or patterns?
\item
  When exploring data, how do you balance curiosity-driven discovery
  with methodological discipline?
\item
  How can EDA findings influence later stages of the data science
  workflow, such as feature engineering, model selection, or evaluation?
\item
  In what ways did EDA help you detect issues of data quality, such as
  missing values or redundancy?
\end{itemize}

EDA is not a one-time step but an iterative mindset that continues
throughout analysis. Revisiting exploratory findings after modeling
often deepens understanding and improves both model performance and
interpretability.

\bookmarksetup{startatroot}

\chapter*{References}\label{references}
\addcontentsline{toc}{chapter}{References}

\markboth{References}{References}

\phantomsection\label{refs}
\begin{CSLReferences}{1}{0}
\bibitem[\citeproctext]{ref-messerli2012chocolate}
Messerli, Franz H. 2012. {``Chocolate Consumption, Cognitive Function,
and Nobel Laureates.''} \emph{N Engl J Med} 367 (16): 1562--64.

\end{CSLReferences}




\end{document}
